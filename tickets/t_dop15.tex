\setcounter{section}{4}
\setcounter{subsection}{15}
\setcounter{equation}{0}
\textbf{\LARGE dop 15. Зависимость pешений диффеpенциальных уpавнений от исходных данных.}

Рассмотрим задачу Коши для дифференциального уравнения первого порядка, разрешенного относительно производной
\begin{equation*}
\begin{split}
y'(t)=f(t,y(t)), \; t\in [t_0-T,t_0+T],\quad y(t_0)=y_0.
\end{split}
\end{equation*}
Пусть функция $f(t,y)$ определена и непрерывна в прямоугольнике
\begin{equation*}
Q=\{(t,y):\: |t-t_0|\leq T,\; A\leq y\leq B\}.
\end{equation*}

\textbf{Теорема 1}\\
Пусть функции $f_1 (t,y)$ и $f_2 (t,y)$ непрерывны в прямоугольнике $Q$ и $f_1 (t,y)$ удовлетворяет условию Липшица по $y$, то есть существует константа  $L > 0$ такая, что
\begin{equation*}
|f_1(t,y)-f_1(t,\widetilde{y})|\leq L|y-\widetilde{y}|,\; \forall(t,y), (t,\widetilde{y})\in Q
\end{equation*}
Тогда, если функции $y_1(t)$ и $y_2(t)$ на отрезке $J = [t_0-T,t_0+T]$ являются решениями задач Коши 
\begin{equation*}
\begin{split}
\begin{cases}
y'_1(t)=f_1(t,y_1(t)),\\
y_1(t_0)=y_{01},
\end{cases}
\:
\begin{cases}
y'_2(t)=f_2(t,y_2(t)),\\
y_2(t_0)=y_{02},
\end{cases}
\end{split}
\end{equation*}
то имеет место неравенство 
\begin{equation*}\tag{1}
\begin{split}
\max_{t \in J} |y_1(t)-y_2(t)|
\leq ( |y_{01}-y_{02}|+T \max_{(t,y)\in Q} |f_1(t,y)-f_2(t,y)|) e^{LT}.
\end{split}
\end{equation*}

$\blacktriangleright\;$ Из леммы об эквивалентности задачи Коши интегральному уравнению следует, что функции $y_1(t)$ и $y_2(t)$ являются решениями интегральных уравнений
\begin{equation*}
\begin{split}
y_1(t)=y_{01}+\int_{t_0}^t f_1(\tau , y_1(\tau))d\tau , \;t\in J =[t_0-T, t_0+T],\\
y_2(t)=y_{02}+\int_{t_0}^t f_2(\tau , y_2(\tau))d\tau , \;t\in J = [t_0-T, t_0+T].
\end{split}
\end{equation*}
Вычитая второе уравнение из первого и оценивая по модулю, имеем
\begin{equation*}
|y_1(t)-y_2(t)|\leq |y_{01}-y_{02}|+\big| \int_{t_0}^t (f_1(\tau , y_1(\tau))- f_2(\tau , y_2(\tau)))d\tau \big|.
\end{equation*}
Вычитая и прибавляя под знаком интеграла $f_1(\tau,y_2(\tau))$, получим
\begin{equation*}\tag{2}
\begin{split}
|y_1(t)-y_2(t)|\leq |y_{01}-y_{02}|+\big| \int_{t_0}^t |f_1(\tau , y_1(\tau))- f_1(\tau , y_2(\tau))|d\tau \big| +\\
+\big| \int_{t_0}^t |f_1(\tau , y_2(\tau))- f_2(\tau , y_2(\tau))|d\tau \big|,\; t\in J.    
\end{split}
\end{equation*}
Учитывая то, что функция $f_1(t,y)$ удовлетворяет условию Липшица, а также оценку
\begin{equation*}
\big| \int_{t_0}^t |f_1(\tau , y_2(\tau))- f_2(\tau , y_2(\tau))|d\tau \big|\leq T \underset{(t,y)\in Q}{max} |f_1(t,y)-f_2(t,y)|,
\end{equation*}
справедливую для всех $t\in J$, неравенство (2) можно переписать так:
\begin{equation*}
\begin{split}
|y_1(t)-y_2(t)|\leq \big( |y_{01}-y_{02}|+  T \underset{(t,y)\in Q}{max} |f_1(t,y)-f_2(t,y)|\big)+\\
+L \big| \int_{t_0}^t |y_1(\tau)- y_2(\tau)|d\tau \big|, \;t\in J.
\end{split}
\end{equation*}
Применив к функции $|y_1(t)-y_2(t)|$ лемму Гронуолла-Беллмана, при $t\in J$ получим неравенство
\begin{equation*}
|y_1(t)-y_2(t)|\leq (|y_{01}-y_{02}|+  T \underset{(t,y)\in Q}{max} |f_1(t,y)-f_2(t,y)|) exp\{L|t-t_0|\},
\end{equation*}
из которого следует оценка (1). Теорема доказана. $\blacksquare$

Рассмотрим прямоугольник
$
Q_+=\{(t,y):\; t_0\leq t\leq t_0+T,\: A\leq y\leq B\}.
$

Далее мы используем следующее простое утверждение из математического анализа, представляющее собой формулу конечных приращений в интегральном виде.

\textbf{Лемма 1}\\
Пусть функция $f(t,y)$ непрерывна в $Q_+$ и имеет в $Q_+$ непрерывную частную производную $f_y(t,y)$. Тогда для любых $(t,y_1), (t,y_2)\in Q_+$ справедливо равенство
\begin{equation*}\tag{3}
f(t,y_1)-f(t,y_2)=\int_0^1 f_y(t,y_2+\theta(y_1-y_2))d\theta(y_1-y_2).
\end{equation*}
Докажем теперь теорему о сравнении решений двух задач Коши, которую также часто называют  неравенством Чаплыгина.

\textbf{Теорема 2. (Теорема сравнения)}\\
Пусть функции $f_1(t,y)$, $f_2(t,y)$ непрерывны в $Q_+$ и $f_1(t,y)$ имеет в $Q_+$ непрерывную частную производную $\frac{\partial f_1}{\partial y}(t,y)$. Тогда, если функции $y_1(t), y_2(t)$ на отрезке $[t_0,t_0+T]$ являются решениями задач Коши
\begin{equation*}
\begin{split}
\begin{cases}
y'_1(t)=f_1(t,y_1(t)),\\
y_1(t_0)=y_{01},
\end{cases}
\:
\begin{cases}
y'_2(t)=f_2(t,y_2(t)),\\
y_2(t_0)=y_{02},
\end{cases}
\end{split}
\end{equation*}
причем
\begin{equation*}
f_1(t,y)\geq f_2(t,y), \;(t,y)\in Q_+, \;y_{01}\geq y_{02},
\end{equation*}
то справедливо неравенство
\begin{equation*}
y_1(t)\geq y_2(t),\; t\in [t_0,t_0+T].
\end{equation*}
$\blacktriangleright\;$ Так как функции $y_1(t)$ и $y_2(t)$ на отрезке $[t_0,t_0+T]$ являются решениями соответствующих уравнений, то они непрерывно дифференцируемы на отрезке $[t_0,t_0+T], A \leq y_i(t) \leq B, i=1,2,$ и справедливо равенство
\begin{equation*}\tag{4}
y'_1(t)-y'_2(t)=f_1(t,y_1(t))-f_2(t,y_2(t)), \;t\in[t_0,t_0+T].
\end{equation*}
Преобразуем правую часть этого равенства, используя формулу конечных приращений (3), опуская для краткости аргумент функций $y_i(t)$:
\begin{equation*}
f_1(t,y_1(t))-f_2(t,y_2(t))= 
f_1(t,y_1)-f_1(t,y_2)+f_1(t,y_2)-f_2(t,y_2)=
\end{equation*}
$$=\int_0^1 \frac{\partial f_1}{\partial y}(t,y_2+\theta(y_1-y_2))d\theta(y_1-y_2)+f_1(t,y_2)-f_2(t,y_2).$$

Введем обозначения
\begin{equation*}
\begin{split}
v(t)=y_1(t)-y_2(t),\\
p(t)= \int_0^1 \frac{\partial f_1}{\partial y}(t,y_2(t)+\theta(y_1(t)-y_2(t)))d\theta,\\
h(t)=f_1(t,y_2(t))-f_2(t,y_2(t)).
\end{split}
\end{equation*}
Тогда $f_1(t,y_1(t))-f_2(t,y_2(t))=p(t)v(t)+h(t)$, и равенство (4) можно переписать так:
\begin{equation*}
v'(t)=p(t)v(t)+h(t),\; t\in[t_0,t_0+T].
\end{equation*}
Решение этого линейного дифференциального уравнения первого порядка с начальным условием $v(t_0)=y_{01}-y_{02}$ имеет вид
\begin{equation*}
v(t)=(y_{01}-y_{02})exp\big\{ \int_{t_0}^t p(\xi)d\xi \big\}+ \int_{t_0}^t exp\big\{\int_{\tau}^t p(\xi)d\xi \big\}h(\tau)d\tau, 
\end{equation*}
$t\in[t_0,t_0+T]$.Так как из условий теоремы следует, что 
\begin{equation*}
y_{01}-y_{02}\geq 0,\; h(t)\geq 0,\; t\in[t_0,t_0+T],
\end{equation*}
то $v(t)=y_1(t)-y_2(t)\geq 0,\; t\in[t_0,t_0+T]$. 
$\blacksquare$\\

Обозначим $Q_\mu=\{(t,y,\mu): \;|t-t_0|\leq T,\; A\leq y\leq B, \;\mu_1\leq \mu\leq \mu_2\}.$

Пусть функция $f(t,y,\mu)$ определена на множестве $Q_\mu$, а функция $y_0(\mu)$ определена на отрезке $[\mu_1,\mu_2]$.

Рассмотрим \textbf{задачу Коши} (ЗК)
\begin{equation*}\tag{5}
y'(t)=f(t,y(t),\mu),\; t\in [t_0-T,t_0+T],
\end{equation*}
\begin{equation*}\tag{6}
y(t_0)=y_0(\mu).
\end{equation*}

\textbf{Теорема 3 (Непрерывная зависимость решения ЗК от параметра)}\\
Пусть функция $f(t,y,\mu)$ непрерывна в $Q_\mu$ и удовлетворяет в $Q_\mu$ условию Липшица по $y$, то есть
\begin{equation*}
|f(t,y_1,\mu)-f(t,y_2,\mu)|\leq L|y_1-y_2|, \;\forall(t,y_1,\mu),(t,y_2,\mu)\in Q_\mu
\end{equation*}
а функция $y_0(\mu)$ непрерывна на отрезке $[\mu_1,\mu_2]$.\\
Тогда если $y(t,\mu)$ - решение задачи Коши (5), (6) на отрезке $[t_0-T,t_0+T]$ для всех $\mu \in [\mu_1,\mu_2]$, то функция $y(t,\mu)$ непрерывна $\mu$ по при $t\in [t_0-T,t_0+T], \mu \in [\mu_1,\mu_2]$.\\

\textbf{Теорема 4 (Дифференцируемость решения ЗК по параметру)}\\
Пусть функция $f(t,y,\mu)$ непрерывна в $Q_\mu$ и имеет в $Q_\mu$ непрерывные частные производные $f_y(t,y,\mu)$, $f_\mu(t,y,\mu)$, а функция $y_0(\mu)$ непрерывно дифференцируема на отрезке $[\mu_1,\mu_2]$.\\
Тогда если $y(t,\mu)$ - решение задачи Коши (5), (6) на отрезке $[t_0-T,t_0+T]$ для всех $\mu \in [\mu_1,\mu_2]$, то функция $y(t,\mu)$ имеет при $t\in [t_0-T,t_0+T], \mu \in [\mu_1,\mu_2]$ производную по $\mu$.\\


% -------- source --------
\bigbreak
[\cite[6-13]{denisov2}]
