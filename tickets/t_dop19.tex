\textbf{\LARGE dop 19. Постановка внешней и внутренней задач Дирихле для уравнения Лапласа. Единственность решения внутренней задачи Дирихле.}

Оператор Лапласа в $\mathbb{R}^n$: $\Delta u \equiv \frac{\partial^2 u}{\partial x_1^2} + \frac{\partial^2 u}{\partial x_2^2} + \ldots + \frac{\partial^2 u}{\partial x_n^2}$.

Уравнение Лапласа: $\Delta u = 0$.\\

\underline{Физическая модель}. Если $\Delta u = 0$ в $\Omega$, то $u(x)$ выражает стационарное распределение
температуры в области $\Omega$. Ясно, что такая температура внутри тела не может превосходить максимального значения температуры на границе $\implies$ \textit{принцип максимума}.\\

\textbf{Принцип максимума.} Пусть $\Omega$ -- ограниченная область, $u \in C^2(\Omega) \cap C(\overline{\Omega}), \; \Delta u = 0$ в $\Omega$. Тогда $\min\limits_{\partial \Omega} u(x) \leq u(x) \leq \max\limits_{\partial \Omega} u(x)$ для $\forall x \in \Omega$. \\

Пусть $\Omega \subset \mathbb{R}^n$ -- произвольная ограниченная область.

\textbf{Внутренняя задача Дирихле:}
\begin{equation*}
    \begin{cases}
    \Delta u = 0 \text{ в } \Omega, \\
    u |_{\partial \Omega} = \varphi.
    \end{cases}
\end{equation*}

Здесь $u = u(x) \in C^2(\Omega) \cap C(\overline{\Omega})$ -- неизвестная функция, $\varphi \in C(\partial \Omega)$ -- заданная функция, $\partial \Omega$ -- граница области $\Omega$.
\\

\textbf{Теорема единственности.} Для 
$\forall \varphi \in C(\partial \Omega)$ внутренняя задача Дирихле имеет не более одного решения в классе $C^2(\Omega) \cap C(\overline{\Omega})$.

\textbf{Доказательство.} Теорема доказывается двумя способами.\\

\textbf{1)} Через 1-ую формулу Грина, но с требованием большей гладкости: $u(x) \in C^2(\overline{\Omega})$. 

От противного. Пусть $u_1(x), \; u_2(x)$ -- два решения внутренней задачи Дирихле и $u(x) = u_2(x) - u_1(x)$.

Запишем 1-ую формулу Грина:

$$\int\limits_{\partial \Omega} u \frac{\partial u}{\partial \nu_y} ds_y = \int\limits_\Omega u \Delta u dx + \int\limits_\Omega \sum_{j=1}^n \left (\frac{\partial u}{\partial x_j} \right )^2 dx.$$

$u|_{\partial \Omega} = 0, \; \Delta u = 0$ в $\Omega$, т.е. слагаемое в левой части и первое слагаемое во второй части равны нулю. Тогда второе слагаемое в правой части тоже должно быть равно нулю.

$\int\limits_\Omega \sum_{j=1}^n \left (\frac{\partial u}{\partial x_j} \right )^2 dx = 0 \implies \sum_{j=1}^n \left (\frac{\partial u}{\partial x_j} \right )^2 = 0 \implies$

$\implies \frac{\partial u}{\partial x_j} \equiv 0 \text{ всюду в } \Omega \quad \forall j = \overline{1,n}.$

$\implies u(x) \equiv C$ в $\overline{\Omega}.\quad u|_{\partial \Omega} = 0 \implies C = 0 \implies u(x) \equiv 0$.\\

\textbf{2)} Используя принцип максимума.

От противного. Пусть $u_1(x), \; u_2(x)$ -- два решения внутренней задачи Дирихле с общей функцией $\varphi \in C(\partial \Omega)$. $u(x) = u_2(x) - u_1(x) \implies$
\begin{equation*}
    \begin{cases}
    \Delta u = 0, \\
    u |_{\partial \Omega} = 0.
    \end{cases}
\end{equation*}
Из принципа максимума следует, что $0 \leq u(x) \leq 0$ для $\forall x \in \Omega \implies u(x) \equiv 0. \quad \blacksquare$
\newline \newline
\textbf{Замечание.} При неограниченной области $\Omega$ вводятся дополнительные условия, связанные с поведением функции на $\infty$.

Пусть $\Omega_0$ -- ограниченная область в $\mathbb{R}^n$. Рассмотрим $\Omega = \mathbb{R}^n \setminus \overline{\Omega}_0$.

В $\Omega_0$ распространена некая физическая субстанция. Требуется изучить ее влияние на внешнее пространство.

Требование регулярности на $\infty$: $u(x) \rightarrow 0$ при $|x| \rightarrow \infty$.

В этом случае единственность переносится и на внешнюю задачу.

% -------- source --------
\bigbreak
[\cite{umf_tix}, файлы 9 и 10.1]
