\setcounter{section}{4}
\setcounter{subsection}{17}
\setcounter{equation}{0}
\textbf{\LARGE dop 17.  Пеpвая кpаевая задача для уpавнения колебаний стpуны. Интегpал энеpгии и единственность pешения пеpвой кpаевой задачи.}

\textbf{\quad I. Уравнение колебаний струны}

\noindent Рассматриваем малые колебания ограниченной натянутой струны.

\noindent\underline{Основные характеристики}:
\begin{itemize}
    \item $x$ -- координата вдоль струны,
    \item $t$ -- время,
    \item $l>0$ -- длина струны,
    \item $\rho>0$ -- линейная плотность струны,
    \item $T>0$ -- натяжение струны (скалярная величина, не зависящая от $x$),
    \item $F(x, t)$ -- линейная плотность внешних сил,
    \item $u(x, t)$ -- отклонение струны (по вертикали) от горизонтального положения ($u \equiv 0$ -- строго горизонтальная <<прямая>> струна, \;$u>0$ -- струна отклонилась вверх, \;$u<0$ -- вниз).
\end{itemize}

\begin{equation}
    u_{tt}(x, t) = a^2 u_{xx}(x, t) + f(x, t), \quad\text{где}\quad a^2 = \frac{T}{\rho(x)}, \quad f(x, t) = \frac{F(x, t)}{\rho}. \label{eq17.1}
\end{equation}
\noindent Уравнение \eqref{eq17.1} называют \textbf{уравнением малых колебаний струны}. Оно выполнено при всех $x \in (0, l)$ и всех $t$ из рассматриваемого интервала $J \subset \mathbb{R}$. 

Обычно $t \geq 0$, и тогда к уравнению \eqref{eq17.1} добавляют два \textit{начальных условия}: $$u(x, 0) = \varphi(x),\;\;\; u_t(x, 0) = \psi(x),$$ т.\,е. задают начальные отклонения и скорости струны (что согласуется со вторым законом Ньютона).

\textbf{\quad II. Энергия колебаний струны}

Найдем выражение для энергии поперечных колебаний струны $E = K + U$, где $K$ -- кинетическая и $U$ -- потенциальная энергии. Элемент струны $dx$, движущийся со скоростью  $v = u_t$, обладает кинетической энергией 
$$
\frac{1}{2} m v^2 = \frac{1}{2} \rho(x) \,dx\, (u_t)^2.
$$
\textbf{Кинетическая энергия} всей струны равна
$$
K = \frac{1}{2}\int\limits_0^l \rho(X) [u_t(x,t)]^2 \, dx.
$$
\textbf{Потенциальная энергия} поперечных колебаний струны, имеющей при $t = t_0$ форму $u(x, t_0) = u_0(x)$, равна работе, которую надо совершить, чтобы струна из положения равновесия перешла в положение $u_0(x)$. Элемент $dx$ под действием равнодействующей сил натяжения $T u_{xx} dx$ за время $dt$ проходит путь $u_t(x, t) dt$. Работа, производимая всей струной за время $t_0$ равна
\begin{equation}\label{eq17.2}
    -\frac{1}{2} \int\limits_0^l T [u_x(x,t_0)]^2 \, dx + \int\limits_0^{t_0} T u_x u_t \bigg|_{0}^l \, dt,
\end{equation}
где $T u_x\big|_{x=0}$ -- величина натяжения на конце струны $x = 0$, $u_t(0, t)\, dt $ -- перемещение этого конца (аналогично для конца $x = l$). То есть второе слагаемое интеграла \eqref{eq17.2} отвечает за работу, которую надо затратить на перемещение концов струны. \\

Если \textbf{концы струны закреплены}, то работа на них будет равна нулю. 
Следовательно, при перемещении закрепленной на концах струны из положения равновесия $u=0$ в положение $u_0(x)$ работа не зависит от способа перевода струны в это положение.
Тогда потенциальная энергия струны в момент времени $t = t_0$ равна работе, взятой с обратным знаком, то есть 
$$
U = \frac{1}{2}\int\limits_0^l T [u_x(x,t_0)]^2 \, dx 
$$
Таким образом, \textbf{полная энергия струны} (\textit{интеграл энергии)} равна
\begin{equation}\label{eq17.3}
    E = \frac{1}{2}\int\limits_0^l \big[T (u_x)^2  + \rho(x) (u_t)^2\big]\, dx 
\end{equation}

\textbf{\quad III. Постановка первой краевой задачи}

\textbf{Первая краевая задача. }Найти функцию $u(x,t)$, определенную в области $0 \leq x \leq l, \; t > 0$, удовлетворяющую уравнению  \eqref{eq17.1} при $0 < x < l, \; t >0$, граничным условиям
$$
u(0, t) = \mu_1(t), \;\; u(l, t) = \mu_2(t), \;\; t> 0,
$$
а также начальным условиям
$$
u(x, 0) = \varphi(x), \;\; u_t(x,0) = \psi(x), \;\; 0<x<l.
$$

\textbf{Задача с закрепленными концами} является важным частным случаем первой краевой задачи при $u(0,t) = u(l, t) = 0, \;\; t> 0$.\\

\textbf{\quad IV. Теорема единственности}

Докажем теорему единственности для уравнения колебаний \eqref{eq17.1} в более общем случае -- когда линейная плотность струны $\rho(x)$ и сила натяжения $T(x)$ могут изменяться. Для этого представим \eqref{eq17.1} в виде 
\begin{equation}\label{eq17.4}
    \rho(x) \frac{\partial ^2 u}{\partial t^2} = \frac{\partial }{\partial x} \left(T(x) \frac{\partial u}{\partial x} \right) + F(x,t),\;\; \rho(x) > 0, \;\; T(x) > 0.
\end{equation}

\textbf{Теорема единственности.} Возможно существование только одной функции $u(x,t)$, определенной в области $0\leq x\leq l, \; t \geq 0$ и удовлетворяющей первой краевой задаче 
$$
\begin{cases}
    \rho(x) \frac{\partial ^2 u}{\partial t^2} = \frac{\partial }{\partial x} \left(T(x) \frac{\partial u}{\partial x} \right) + F(x,t), \;\; 0< x< l,\; t > 0,\\[2mm]
    u(x, 0) = \varphi(x), \;\; u_t(x,0) = \psi(x), \;\; 0< x< l,\\[2mm]
    u(0, t) = \mu_1(t), \;\; u(l, t) = \mu_2(t), \;\; t> 0,
\end{cases}
$$
если выполнены следующие условия:
\begin{enumerate}
    \item функция $u(x,t)$ и производные, входящие в уравнение \eqref{eq17.4}, а также производная $u_{xt}$ непрерывны на отрезке $0 \leq x \leq l$ при $t \geq 0$;
    \item коэффициенты $\rho(x)$ и $T(x)$ непрерывны на отрезке $0 \leq x \leq l$.
\end{enumerate}
$\blacktriangleright\;$ Докажем от противного. 
Допустим, что существует два решения рассматриваемой задачи: $u_1(x,t), \; u_2(x,t)$, и рассмотрим разность $v(x,t) = u_1(x,t) - u_2(x,t)$. 

Функция $v(x,t)$ удовлетворяет задаче с однородными условиями и отсутствием внешних сил
$$
\begin{cases}
    \rho(x) \frac{\partial ^2 v}{\partial t^2} = \frac{\partial }{\partial x} \left(T(x) \frac{\partial v}{\partial x} \right), \;\; 0< x< l,\; t > 0,\\[2mm]
    v(x, 0) = v_t(x,0) = 0, \;\; 0< x< l,\\[2mm]
    v(0, t) = v(l, t) = 0 \;\; t> 0,
\end{cases}
$$
а также условию 1) теоремы единственности. Докажем, что $v(x,t) \equiv 0$. 

Интеграл энергии \eqref{eq17.3} в момент времени $t$ для струны, колебания которой описывает функция $v(x,t)$, имеет вид
$$
 E(t) = \frac{1}{2}\int\limits_0^l \big[T (v_x)^2  + \rho (v_t)^2\big]\, dx. 
$$
Продифференцируем $E(t)$ по $t$:
$$
\frac{d E(t)}{dt} = \int\limits_0^l (T v_x v_{xt} + \rho v_t v_{tt}) \, dx.
$$
Интегрируя по частям первой слагаемое право части, получим:
$$
\int\limits_0^l T v_x v_{xt} \, dx = \big[ T v_x v_t\big]\bigg|_0^l - \int\limits_0^l v_t (T v_x)_x \, dx.
$$
Подстановка обращается в нуль в силу граничных условий: $v(0, t) = 0 \implies v_t(0,t) = 0$. Отсюда 
$$
\frac{d E(t)}{dt} = \int\limits_0^l \big[\rho v_t v_{tt} - v_t (T v_x)_x \big]\,dx = \int\limits_0^l v_t \big[\rho v_{tt} - (T v_x)_x\big]\, dx = 0, 
$$
т.е. $E(t) = const$. Учитывая начальные условия, получаем 
$$
E(t) = E(0) = \frac{1}{2} \int\limits_0^l \big[ T (v_x)^2 + \rho (v_t)^2\big]\bigg|_{t=0}\,dx = 0,
$$
так как $v(x,0) = v_t(x, 0) = 0$. 

По условию $T > 0, \; \rho >0 \implies v_x(x,t) \equiv 0, \; v_t(x,t) \equiv 0$, откуда следует тождество $v(x,t) = const = C_0$. 

Из начального условия находим $v(x, 0) = C_0 = 0 \implies v(x,t) \equiv 0$. 

Таким образом показано, что если существуют две функции $u_1(x,t),\;u_2(x,t)$, удовлетворяющие всем условиям теоремы, то $u_1(x,t) \equiv u_2(x,t)$. $\;\blacksquare$

 % -------- source --------
\bigbreak
[\cite{umf_tix}, файл 12]

[\cite{urmati_tikhonov}, pages 28-30, 42, 46-48]