\textbf{\LARGE dop.16 Постановка ваpиационных задач. Необходимые условия экстpемума.}

Рассмотрим множество $M \subset C[x_0, x_1]$.


\textbf{Определение 1.}
Функционалом называется отображение множества $M$ в $\mathbb{R}$.

\textbf{Определение 2.}
Допустимой вариацией функции $y_0(x) \in M$ называется $\forall$ функция $\delta y(x):y_0(x) + \delta y(x) \in M$.

\textbf{Определение 3.}
Вариацией $\delta\Phi[y_0(x), \delta y(x)]$ функционала
$\Phi[y(x)]$ на функции $y_0(x) \in M$ наз: $\frac{d}{dt}\Phi[y_0(x)+t\delta y(x)]\Big|_{t=0}$

\textbf{Определение 4.}
Функционал $\Phi[y(x)]$ достигает на функции
$y_0(x) \in M$ глобального минимума (максимума) на множестве $M$, если для $\forall$ $y(x) \in M$ выполнено неравенство $\Phi[y_0(x)]\leq\Phi[y(x)]  (\Phi[y_0(x)]\geq\Phi[y(x)])$.


$\sqsupset$ на  $M$ введена норма функции $y(x)$, например:
$\norm{y(x)}=\displaystyle\max_{x_0\leq x \leq x_1} |y(x)|$


\textbf{Теорема 1.(необходимое условие экстремума)}

    Если функционал $\Phi[y(x)]$ достигает на функции
    $y_0(x) \in M$ локального максимума или минимума на множестве $M$ и
    вариация функционала на $y_0(x)$ $\exists$, то вариация функционала $\delta\Phi[y_0(x), \delta y(x)]$ равна нулю для $\forall$ допустимой вариации $\delta y(x)$.
    
$\blacktriangleright\;$
    $\sqsupset$ функционал $\Phi[y(x)]$ достигает на функции
    $y_0(x)$ локального экстремума. Рассмотрим $\Phi[y_0(x) + t\delta y(x)]$, где $\delta y(x)$
    произвольная вариация $y_0(x)$. При фиксированных $y_0(x)$ и $\delta y(x)$ функционал 
    $\Phi[y_0(x) + t\delta y(x)]$ является функцией переменной t:
    $\varphi(t) = \Phi[y_0(x) + t\delta y(x)].$
    Т.к  функционал $\Phi[y(x)]$ достигает на функции $y_0(x)$ локального
    экстремума, то у функции $\varphi(t)$ точка $t=0$ является точкой локального экстремума. $\Longrightarrow$ если $\varphi'(0)$ существует, то
    $\varphi'(0)=0$. $\exists$ $\varphi'(0)$ следует из $\exists$ вариации функционала $\Phi[y(x)]$ на $y_0(x)$: $\frac{d}{dt}\varphi(t)\Big|_{t=0}=\frac{d}{dt}\Phi[y_0(x)+t\delta y(x)]\Big|_{t=0} \Longrightarrow$
    $\delta \Phi[y_0(x), \delta y(x)]=\frac{d}{dt}\Phi[y_0(x)+t\delta y(x)]\Big|_{t=0}=0 \text{ для } \forall \delta y(x)$$\blacksquare$.


$\sqsupset$ $C^n_0 [x_0, x_1]$ -- множество функций $y(x) \in C^n[x_0, x_1]:$ $y^{(m)}(x_0)=y^{(m)}(x_1)=0, m=0..n-1.$

\textbf{Лемма 1 (Основная лемма вариационного исчисления)}

    $\sqsupset$ $f(x)\in С[x_0, x_1]$ функция: 
    $ \int_{x_0}^{x_1} f(x)y(x) \,dx=0$
    для $\forall$ $y(x) \in C^n_0[x_0, x_1]$. Тогда $f(x) \equiv 0$ на отрезке $[x_0, x_1]$.
    
\rule{275pt}{0.5pt} \\
\textbf{\large Уравнение Эйлера}

Рассмотрим множество $M$ непрерывно дифференцируемых на
$[x_0, x_1]$ функций $y(x):$ $y(x_0) = y_0, y(x_1) = y_1$. Определим
на $M$ функционал: $\Phi[y(x)]=\int_{x_0}^{x_1}F(x,y(x),y'(x))\,dx$ (1).

\textbf{Теорема 2.(необходимое условие экстремума)}

    $\sqsupset$ при $x\in[x_0,x_1], (y, p) \in \mathbb{R}^2$ у
    функции $F(x,y,p)$ $\exists$ непрерывные вторые част производные. Если функционал (1) достигает локального экстремума на функции $y_0(x)\in M$, имеющей непрерывную вторую производную на отрезке $[x_0, x_1]$, то функция $y_0(x)$ является решением дифференциального ур-я: $F_y(x, y(x), y'(x))-\frac{d}{dx}F_p(x,y(x),y'(x))=0, x_0\leq x\leq x_1.$(2)
    
$\blacktriangleright\;$
    Найдем вариацию функционала (1) на $y_0(x)$. Из
    определения множества $M$ следует, что допустимой вариацией $\delta y(x)$
    функции $y_0(x)$ является $\forall$ непрерывно дифференцируемая на отрезке $[x_0, x_1]$ функция, обращающаяся в ноль на концах этого отрезка. То есть $\delta y(x) \in C^1_0\in [x_0, x_1]$. Используя определение вариации функционала, получим

$        \delta\Phi[y_0(x),\delta y(x)]=\frac{d}{dt}\Phi[y_0(x)+t\delta y(x)]\Big|_{t=0}=\frac{d}{dt}\int_{x_0}^{x_1} F(x, y_0(x) + t\delta y(x), y'_0(x) + t(\delta y)'(x))dx\Big|_{t=0}=
        \\
        \int_{x_0}^{x_1}\Big\{F_y(x,y_0(x)+t\delta y(x), y'_0(x) + t(\delta y)'(x))\delta y(x)+
        F_p(x,y_0(x)+t\delta y(x), y'_0(x) + t(\delta y)'(x))(\delta y)'(x)\Big\}dx\Big|_{t=0}=
        \\
        \int_{x_0}^{x_1} \Big\{F_y(x, y_0(x), y'_0(x))\delta y(x) + F_p(x, y_0(x), y'_0(x))(\delta y)'(x)\Big\}dx$

    Из теоремы о необходимом условии экстремума $\Longrightarrow$ что вариация
    функционала на $y_0(x)$ должна равняться нулю, то есть:
    $\int_{x_0}^{x_1} F_y(x, y_0(x), y'_0(x))\delta y(x)dx+\int_{x_0}^{x_1} F_p(x, y_0(x), y'_0(x))(\delta y)'(x)dx=0$
    Интегрируя по частям второй интеграл и учитывая то, что $\delta y(x_0)=\delta y(x_1)=0$, получим:
    $\int_{x_0}^{x_1}\Big\{F_y(x, y_0(x), y'_0(x))-\frac{d}{dx}F_p(x, y_0(x), y'_0(x))\Big\}\delta y(x)dx=0$
    Это равенство выполняется для $\forall$ функции $\delta y(x) \in C^1_0 [x_0, x_1]$. Применяя  лемму 1, имеем
    $F_y(x, y_0(x), y'_0(x))-\frac{d}{dx}F_p(x, y_0(x), y'_0(x))=0,  x_0\leq x\leq x_1$
    $\Longrightarrow$ функция $y_0(x)$ является решением ур-я (2)$\blacksquare$.

Уравнение (2) называется уравнением Эйлера для функционала (1).

\rule{275pt}{0.5pt} \\
\textbf{\large Функционал, зависящий от производных порядка выше
первого}

Рассмотрим множество $M$ функции $y(x) \in C^n[x_0, x_1]:$
$y(x_0) = y_{00} , y'(x_0) = y_{01}, y''(x_0) = y_{02},..., y^{(n-1)}(x_0) = y_{0n-1},$
$y(x_1) = y_{10} , y'(x_1) = y_{11}, y''(x_1) = y_{12},..., y^{(n-1)}(x_1) = y_{1n-1}$
Определим на $M$ функционал:$\Phi[y(x)]=\int_{x_0}^{x_1}F(x,y(x),y'(x),...,y^{(n)}(x))dx$(3), где $F(x,y,p_1,...,p_n)$ определена и непрерывна при $x \in [x_0, x_1]$,
$(y,p_1,..., p_n) \in \mathbb{R}^{n+1}$.

\textbf{Теорема 3.(необходимое условие экстремума)}
    $\sqsupset$ функция $F(x,y,p_1,...,p_n)$ имеет при $x\in[x_0, x_1]$,$(y,p_1,..., p_n) \in \mathbb{R}^{n+1}$ непрерывные частные производные порядка $2n$. Если  $\bar y(x) \in M$, $\bar y(x) \in C^{2n}[x_0, x_1]$, и на ней достигается экстремум функционала (3) на $M$ , то $\bar y(x)$ является решением уравнения:
    $F_y-\frac{d}{dx}F_{p_1}+...+(-1)^n\frac{d^n}{dx^n}F_{p_n}=0, x_0 \leq x \leq x_1,$
    где $F = F(x,y(x),y'(x),..., y^{(n)}(x))$.
    
$\blacktriangleright\;$
    В силу необход.усл.экстр вариация
    функционала (3) на функции $\bar y(x)$ должна обращаться в 0 для
    $\forall$ допустимой вариации $\delta y(x) \in C^n_0[x_0, x_1]$.
    По определению вариации функционала имеем:
    $\delta\Phi[\bar y(x), \delta y(x)]=\frac{d}{dt}\Phi[\bar y(x)+t\delta y(x)]\Big|_{t=0}=$
    $\frac{d}{dt}\int_{x_0}^{x_1}F(x,\bar y(x) + t\delta y(x), \bar y'(x)+t(\delta y)'(x),...,\bar y^{(n)}(x)+t(\delta y)^{(n)}(x))dx\Big|_{t=0}$
    Дифференцируем интеграл по параметру $t$, полагая затем $t = 0$ и приравнивая вариацию к 0, получим:
    $\int_{x_0}^{x_1}(F_y\delta y(x) + F_{p_1}(\delta y)'(x)+...+F_{p_n}(\delta y)^{(n)}(x))dx=0$
    Интегрируя по частям и учитывая то, что функция $\delta y(x)$ и ее производные обращаются в 0 на концах отрезка, имеем
    $\int_{x_0}^{x_1} (F_y-\frac{d}{dx}F_{p_1}+...+(-1)^n\frac{d^n}{dx^n}F_{p_n})\delta y(x) dx=0$
    Т.к это равенство выполнено для $\forall$ функции $\delta y(x) \in C^n_0[x_0, x_1]$,
    то, применяя лемму 1, получим, что
    функция $\bar y(x)$ является решением дифференциального ур-я (3)$\blacksquare$

\rule{275pt}{0.5pt} \\
\textbf{\large Функционал, зависящий от функции двух переменных}


Рассмотрим функционал:\\
$ \Phi[u(x, y)]=\iint_D F(x, y, u(x, y), u_x(x, y), u_y(x, y))dxdy$(4),
где $F(x, y, u, p, q)$ – заданная функция, а $D$ – область, ограниченная контуром $L$. $\sqsupset$ функция $F(x, y, u, p, q)$ имеет непрерывные 2 частные производные при $(x, y) \in \bar D = D \bigcup L, (u, p, q) \in \mathbb{R}^3$. $\sqsupset$ $M$ – множество функции $u(x, y)$, имеющих в $\bar D$ непрерывные частные производные и принимающих на $L$ заданные значения $u(x, y) = \varphi(x, y),(x, y) \in L$. Вариация функции $u(x, y)$, не выводящая ее из $M$, – это функция $\delta u(x, y)$, имеющая в $\bar D$ непрерывные частные производные и $\delta u(x, y) = 0,(x, y) \in L$.

\textbf{Лемма 2 (аналог леммы вариационного исчисления)}
    $\sqsupset$ функция $f(x, y)$ непрерывна в $\bar D$. Если
    $\iint_D f(x,y)v(x,y)dxdy=0$
    для $\forall$ функции $v(x, y)$, имеющей непрерывные част производные в $\bar D$ и обращающейся в 0 на контуре $L$, то $f(x, y) = 0, (x, y) \in D$.

\textbf{Теорема 4.(необходимое условие экстремума)}
    $\sqsupset$, что функция $F(x, y, u, p, q)$ имеет
    непрерывные 2 част производные при $(x, y) \in \bar D, (u, p, q) \in \mathbb{R}^3$.
    Если экстремум функционала (4) достиг на функции $\bar u(x, y) \in M$, имеющей непрерывные 2 частные производные в $\bar D$, то эта функция является решением ур-я в част производных.$F_u-\frac{\partial F_p}{\partial x}-\frac{\partial F_q}{\partial y}=0, (x,y)\in D.$(5)
    
\rule{275pt}{0.5pt} \\
\textbf{\large Вариационная задача на условный экстремум}

Рассмотрим 2 функционала: $\Phi[y(x)]=\int_{x_0}^{x_1} F(x,y(x),y'(x))dx$(7) и \\ $\Psi[y(x)]=\int_{x_0}^{x_1} G(x,y(x),y'(x))dx$(8),
где $F(x, y, p), G(x, y, p)$ – заданные дважды непрерывно дифференцируемые функции своих аргументов.
Рассмотрим следующую экстремальную задачу. $\sqsupset$ требуется найти функцию $\bar y(x)$, на которой достигается экстремум функционала (7) на:
$M_\Psi=\{y(x)\in C^1[x_0, x_1] : y(x_0) = y_0, y(x_1) = y_1, \Psi[y(x)]=l\}$
Т.е, нужно найти экстремум функционала (7) на
множестве функции определяемом тем условием, что функционал (8)
принимает на этом множестве const значение. Вариационные задачи такого типа называются \textbf{задачами на условный экстремум}.

\textbf{Теорема 5.(необходимое условие экстремума)}
    $\sqsupset$ на функции $\bar y(x) \in M_\Psi, \bar y(x) \in C^2[x_0, x_1]$,
    достигается экстремум функционала (7) на множестве $M_\Psi$. Если
    $\exists$ функция $\delta y_0(x) \in C^1[x_0, x_1], \delta y_0(x_0) = \delta y_0(x_1) = 0:$ вариация $\delta\Psi[\bar y(x), \delta y_0(x)] \neq 0$, то найдется число $\lambda:$ $\bar y(x)$ удовлетворяет ур-ю: $L_y(x, y(x), y'(x))-\frac{d}{dx}L_p(x, y(x), y'(x))=0, x_0 \leq x \leq x_1$(9), где $L(x, y, p) = F(x, y, p) + \lambda G(x, y, p)$ (10).
    
[\cite{denisov2}]