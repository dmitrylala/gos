\setcounter{section}{15}
\setcounter{subsection}{10}
\setcounter{equation}{0}
\textbf{\LARGE osn 10. Прямая и плоскость, их уравнения. Взаимное расположение прямой и плоскости,  основные задачи на прямую и плоскость.}

Если в пространстве $V_3$ зафиксированы точка $O$ и базис $\{e_1, e_2, e_3\}$, то говорят что в пространстве задана \textbf{афинная система координат} (или \textbf{общая декартова система координат}) $\{O, e_1, e_2, e_3\}$. Точка $O$ называется \textbf{началом координат}. Оси, проходящие через начало координат и определенные векторами $\{e_1, e_2, e_3\}$, называются \textbf{осями координат}. (Обозначается как $O_{xyz}$)

Ненулевой вектор коллинеарный прямой называется ее \textbf{направляющим вектором}: $\overline{M_0M} = t \overline{a}$   

Два неколлинеарных вектора, параллельных плоскости, называются ее \textbf{направляющими векторами}. 


\bigbreak
\centerline{\textbf{Канонические уравнения.}}

\textbf{Уравнение прямой}. На плоскости в афинной системе координат $O_{xy}$ уравнение прямой \textit{l}, проходящей через точку $M_0(x_0,y_0)$ с направляющим вектором $\overline{a}=\{l,m\}$ имеет вид:
$$\begin{vmatrix} x-x_0 & y-y_0 \\ l & m \end{vmatrix} = 0 \iff \frac{x-x_0}{l} = \frac{y-y_0}{m} $$
\begin{proof}
$\mathLet M(x,y)$ -- точка. Тогда $\overline{M_0M}$ = $(x - x_0, ~ y - y_0)$. Условие $M_0M = ta$ в силу лин. координат означает, что в определении первая строка линейно выражена через вторую, а это равносильно определителю (выше). Равенство нулю определителя второго порядка равносильно пропорциональности его строк.
\end{proof}


\textbf{Уравнение плоскости}. 
В пространстве в афинной системе координат $O_{xyz}$ уравнение плоскости $\pi$, проходящей через точку $M_0(x_0,y_0,z_0)$ с направляющими векторами $\overline{p_i}=\{l_i,m_i,k_i\}$ $(i=1,2)$ имеет вид:
$$\begin{vmatrix} x-x_0 & y-y_0 & z-z_0\\ l_1 & m_1 & k_1 \\ l_2 & m_2 & k_2 \end{vmatrix} = 0 $$

\bigbreak
\centerline{\textbf{Параметрические уравнения.}}

\mathLet \ r = $\overline{OM}$, $r_0 = \overline{OM_0}$ -- радиус-векторы точек M и $M_0$ относительно полюса $O$. Если $\overline{a}$ - направляющий вектор, то $\overline{M_0M} = t\overline{a},~t\in\mathbb{R}$. Тогда $\overline{M_0M}$ = $r-r_0$ может быть записано в виде:

\textbf{Уравнение прямой}, проходящей через точку $M_0(\overline{r}_0)$ с направляющим вектором $\overline{a}=\{l,m\}$ имеет вид:
$$\overline{r} = \overline{r}_0 + \overline{a}t,~t\in\mathbb{R}$$

\textbf{Уравнение плоскости}, проходящей через точку $M_0(\overline{r}_0)$ с направляющими векторами $\overline{p}_i=\{l_i,m_i,k_i\}$, $(i=1,2)$ имеет вид:
$$\overline{r} = \overline{r}_0 + u\overline{p}_1 + v\overline{p}_2,~u,v\in\mathbb{R}$$

\bigbreak
\centerline{\textbf{Общие уравнения.}}

\textbf{Теорема}. Линия на плоскости (или поверхность в пространстве) --- есть прямая (плоскость) $\iff$ она является алгебраической линией (поверхностью) первого порядка, т. е. задается уравнением $$Ax+By+C=0,~A^2+B^2\neq0$$ $$(Ax+By+Cz+D=0,~A^2+B^2+C^2\neq0~\text{соответственно})$$ 
Это уравнение называется \textbf{общим уравнением} прямой на плоскости (плоскости в пространстве).

\begin{proof} $(\implies)$
$l$ --- прямая, каноническое уравнение прямой $\begin{vmatrix} x-x_0 & y-y_0 \\ l & m \end{vmatrix} = 0 \implies mx -ly -mx_0 +ly_0 = 0,~A=m,~B=-l,~C=-mx_0 +ly_0$, при этом $A^2+B^2\neq0$, так как $\overline{a}\neq 0$.

$(\impliedby)$ Рассмотрим линию $l:~Ax+By+C=0,~A^2+B^2\neq0$. Заметим, что при $x_0=-\frac{AC}{A^2+B^2},~y_0=-\frac{BC}{A^2+B^2}$ точка $M_0(x_0,y_0)\in l$. Вычитая из уравнения линии $Ax+By+C=0$ уравнение $Ax_0+By_0+C=0$, получим $A(x-x_0)+B(y-y_0)=0$, а значит $\begin{vmatrix} x-x_0 & y-y_0 \\ -B & A \end{vmatrix} = 0$ --- каноническое уравнение прямой, проходящей через точку $M_0$, с направляющим вектором $\{-B,A\}$, а значит линия $l$ --- прямая. 
\end{proof}
    


\textbf{Взаимное расположение прямой и плоскости.} 

\mathLet \ плоскость $\pi$ задана общим уравнением $Ax + By + Cz + D = 0$, а прямая $l$ задана каноническим уравнением $\frac{x-x_0}{m} = \frac{y-y_0}{n} = \frac{z-z_0}{k}$, $M(x_0, y_0, z_0) \in l$, $\overline{p} = (A, B, C)$ -- вектор нормали. Тогда:
\begin{enumerate}
    \item Прямая $l$ принадлежит плоскости $\pi$ $\iff$ \newline
    $
    \begin{cases}
        M \in \pi \implies Ax_0 +By_0 +Cz_0 +D = 0 \\
        (l, \overline{p}) = 0  \implies Am+Bn+Ck = 0
    \end{cases}
    $
    \item Прямая $l$ параллельна плоскости $\pi$ $\iff$ \newline
    
    $\begin{cases}
        M \notin \pi \implies Ax_0+By_0+Cz_0+D\neq 0 \\
        (l, \overline{p}) = 0  \implies Am+Bn+Ck = 0
    \end{cases}$
    
    \item Прямая $l$ перпендикулярна плоскости $\pi$ $\iff$ \newline
    $l \| \overline{p} \implies$ $\frac{A}{m} = \frac{B}{n} = \frac{C}{k}$.
    
    \item Угол $\varphi$ между прямой $l$ и плоскостью $\pi$: \newline
    $$  sin\varphi = sin(l, \pi) = \frac{| (l, \overline{p}) |}{|l| \cdot |\overline{p}|} = \frac{|Am+Bn+Ck|}{\sqrt{A^2+B^2+C^2}\cdot\sqrt{m^2+n^2+k^2}}$$
\end{enumerate}
   
\textbf{Основные задачи на прямую и плоскость.}

\textbf{Углом между двумя прямыми в пространстве} называется любой из углов между параллельными им прямыми, проходящими через какую-либо точку пространства. Таким образом, две прямые в пространстве образуют между собой два различных угла в сумме равных $\pi$. Угол между направляющими вектоами прямых равен одному из этих углов. Угол между прямыми $l_i: r = r_i + ta_i, i = 1,2$, совпадающий с угом между направляющими векторами $a_i = \{m_i, n_i, k_i\}$:

$$cos\varphi = \frac{m_1m_2 + n_1n_2 + k_1k_2}{\sqrt{m_1^2 + n_1^2 + k_1^2}\sqrt{m_2^2 + n_2^2 + k_2^2}}$$

\textbf{Углом между прямой и плоскостью} (если они не перпендикулярны) называется меньший из углов между этой прямой и ее ортогональной проекцией на плоскость. 
Если же прямая и плоскость перпендикулярны, то угол между ними считается равным $\pi / 2$. 
Угол $\varphi$ между прямой $l: r = r_0 + ta$ и плоскостью $\pi: Ax + By + Cz + D = 0$ находится как
дополнительный к углу между направляющим вектором прямой $\overline{a} = \{m, n, k\}$ 
и вектором нормали к плоскости $\overline{n} = \{A, B, C\}$ и вычисляется:

$$sin\varphi = \frac{|Am + Bn + Ck|}{\sqrt{m^2 + n^2 + k^2}\sqrt{A^2 + B^2 + C^2}}, ~ 0 \leq \varphi \leq \pi / 2$$

\textbf{Расстояние $\rho(M_1, l)$ от точки $M_1(r_1)$} до прямой $l: r = r_0 + ta$ находится как высота h параллелограмма, построенного на векторах a и $\overline{M_0M_1}$ площадь и основание которого известны

$$\rho(M_1, l) = \frac{|[a, r_1 - r_0]|}{|a|}$$

\textbf{Расстоянием между скрещивающимися прямыми} $l_i: r = r_i + ta_i, ~ i = 1,2$ называется расстояние между параллельными плоскостями, в которых лежат прямые $l_1, l_2$. Это расстояние $\rho(l_1, l_2)$ находится как высота параллелепипеда, построенного на векторах $\overline{M_1M_2}, a_1, a_2$, объем и площадь основания которого известны:

$$\rho(l_1, l_2) = \frac{|(r_2 - r_1, a_1, a_2)|}{|[a_1, a_2]|}$$

\textbf{Найти уравнение прямой $l$:} $l$  проходит через т. $M_0(x_0,y_0,z_0)$ и перпендикулярна плоскости $Ax+By+Cz+D=0$. Ответ: $\frac{x-x_0}{A} = \frac{y-y_0}{B} = \frac{z-z_0}{C}$.

\textbf{Найти уравнение плоскости $\pi$:} $\pi$ проходит через т. $M_0(x_0,y_0,z_0)$ и перпендикулярна прямой $\frac{x-x_1}{m} = \frac{y-y_1}{n} = \frac{z-z_1}{k}$. Ответ: $m(x-x_0)+n(y-y_0)+k(z-z_0) = 0$.

\textbf{Найти уравнение плоскости $\pi$:} $\pi$ проходит через прямую $\frac{x-x_1}{m} = \frac{y-y_1}{n} = \frac{z-z_1}{k}$ и через т. $M_0(x_0,y_0,z_0)$, не лежащую на этой прямой. Ответ: $$\begin{vmatrix} x-x_0 & y-y_0 & z-z_0\\ x_1-x_0 & y_1-y_0 & z_1-z_0 \\ m & n & k \end{vmatrix} = 0 $$

\textbf{Найти уравнение плоскости $\pi$:} $\pi$ проходит через прямую $\frac{x-x_1}{m_1} = \frac{y-y_1}{n_1} = \frac{z-z_1}{k_1}$ и параллельна другой данной прямой $\frac{x-x_2}{m_2} = \frac{y-y_2}{n_2} = \frac{z-z_2}{k_2}$ (две данные прямые не параллельны). Ответ: $$\begin{vmatrix} x-x_1 & y-y_1 & z-z_1\\ m_1 & n_1 & k_1 \\ m_2 & n_2 & k_2 \end{vmatrix} = 0 $$

\textbf{Найти уравнение плоскости $\pi$:} $\pi$ проходит через две данные точки $M_0(x_0,y_0,z_0)$ и $M_1(x_1,y_1,z_1)$ и  перпендикулярна данной плоскости $Ax + By + Cz + D = 0$ (прямая $M_0M_1$ и данная плоскость не перпендикулярны). Ответ: $$\begin{vmatrix} x-x_0 & y-y_0 & z-z_0\\ x_1-x_0 & y_1-y_0 & z_1-z_0 \\ A & B & C \end{vmatrix} = 0 $$