\setcounter{section}{4}
\setcounter{subsection}{13}
\setcounter{equation}{0}
\textbf{\LARGE dop 13. Функция Грина первой краевой задачи для обыкновенного дифференциального уравнения  второго порядка. Условия существования решения краевой задачи.}

Рассмотрим краевую задачу:
\begin{equation} \label{eq13.1}
\begin{cases}
Ly \equiv \frac{d}{dx} \left ( p(x) \frac{dy}{dx} \right ) - q(x) y = f(x), \quad 0 \leq x \leq l, \\
\alpha_1 y'(0) + \beta_1 y(0) = 0, \\
\alpha_2 y'(l) + \beta_2 y(l) = 0,
\end{cases}
\end{equation}

где $p(x) \in C^1[0,l], \; p(x) > 0; \;\; q(x), f(x) \in C[0,l];\; \; \alpha_i^2 + \beta_i^2 > 0, i = 1, 2$.

В случае $f(x) \equiv 0$ задача называется однородной.

\underline{Определение}. Функция $y(x)$ наз. решением краевой задачи \eqref{eq13.1}, если $y(x) \in C^2[0,l]$ и удовлетворяет \eqref{eq13.1}.

\underline{Определение}. \textbf{Функцией Грина} задачи \eqref{eq13.1} наз. определенная в квадрате $[0,l] \times [0,l]$ функция $G(x, \xi)$, которая удовлетворяет следующим условиям:

\begin{enumerate}
    \item для $\forall \xi \in (0,l)$ функция $G(x, \xi)$ дважды непрерывно дифференцируема по $x$ на $[0, \xi) \cup (\xi, l]$ и удовлетворяет однородному уравнению: $$\frac{d}{dx} \left ( p(x) \frac{d G(x, \xi)}{dx} \right ) - q(x) G(x, \xi) = 0, \; 0 \leq x \leq l, \; x \ne \xi;$$
    \item функция $G(x, \xi)$ удовлетворяет однородным краевым условиям по $x$: $$\alpha_1 G_x(0, \xi) + \beta_1 G(0, \xi) = 0, \; \alpha_2 G_x(l, \xi) + \beta_2 G(l, \xi) = 0, \quad \forall \xi \in (0, l);$$
    \item функция $G(x, \xi)$ непрерывна в квадрате $[0, l] \times [0, l]$, а частная производная $G_x(x, \xi)$ имеет в точке $x = \xi$ разрыв первого рода с величиной скачка $$\frac{d}{dx} G(x, \xi) \big |_{x = \xi + 0} - \frac{d}{dx} G(x, \xi) \big |_{x = \xi - 0} = \frac{1}{p(\xi)}.$$
\end{enumerate}

\textbf{Теорема.} Если однородная краевая задача $L u = 0, \; \alpha_1 u'(0) + \beta_1 u(0) = 0, \; \alpha_2 u'(l) + \beta_2 u(l) = 0$ имеет только тривиальное решение, то функция Грина задачи \eqref{eq13.1} существует и единственна.

\textbf{Доказательство.} Построим два линейно независимых решения $u_1, \; u_2$ однородного уравнения, каждое из которых удовлетворяет только одному из граничных условий:
\begin{equation*}
    \begin{cases}
        L u_1 = 0, \quad 0 \leq x \leq l,\\
        u_1(0) = -\alpha_1,\\
        u_1'(0) = \beta_1.
    \end{cases}
    \begin{cases}
        L u_2 = 0, \quad 0 \leq x \leq l,\\
        u_2(l) = -\alpha_2,\\
        u_2'(l) = \beta_2.
    \end{cases}
\end{equation*}

Они линейно независимы, т.к. в противном случае однородная краевая задача имела бы ненулевое решение.

Функцию Грина будем искать в виде: $$G(x, \xi) = 
\begin{cases} 
c_1(\xi) u_1(x), \quad 0 \leq x \leq \xi, \\
c_2(\xi) u_2(x), \quad \xi \leq x \leq l.
\end{cases}$$

$c_1, c_2$ определяются из условия непрерывности $G(x, \xi)$ и разрыва $G_x(x, \xi)$ в точке $x = \xi$:

\begin{equation*}
    \begin{cases}
        c_1(\xi) u_1(\xi) = c_2(\xi) u_2(\xi), \\
        c_2(\xi) u_2'(\xi) - c_1(\xi) u_1'(\xi) = \frac{1}{p(\xi)}.
    \end{cases}
\end{equation*}

Получили систему уравнений относительно двух неизвестных функций $c_1(\xi), \; c_2(\xi)$, решая которую получим:
$$c_1(\xi) = \frac{u_2(\xi)}{W(\xi) p(\xi)}, \quad c_2(\xi) = \frac{u_1(\xi)}{W(\xi) p(\xi)},$$

где $W(\xi) = u_1(\xi) u_2'(\xi) - u_2(\xi) u_1'(\xi)$ -- определитель Вронского.\\

Получили окончательную формулу для функции Грина: 
\begin{equation} \label{eq13.2}
    G(x, \xi) = \frac{1}{W(\xi) p(\xi)} \cdot  
\begin{cases}
    u_2(\xi) u_1(x), \;\; 0 \leq x \leq \xi, \\
    u_1(\xi) u_2(x), \;\; \xi \leq x \leq l.
\end{cases}
\end{equation}

Докажем единственность. Пусть $\exists$ две функции Грина $G(x, \xi),\; \widehat{G}(x, \xi)$. Пусть $\xi$ -- произвольная точка интервала $(0,l) \implies z(x) = G(x, \xi) - \widehat{G}(x, \xi)$ -- непрерывна на $[0,l]$ и имеет непрерывную производную $z'(x)$ на $[0,l]$, т.к. у $G_x(x, \xi)$ и $\widehat{G}_x(x, \xi)$ разрыв в точке $x = \xi$ одинаков.\\

$Lz = 0, \; x \ne \xi \implies z'' = \frac{q(x) z(x) - p'(x) z'(x)}{p(x)}$ -- непрерывна при $x = \xi \implies Lz = 0, \; 0 \leq x \leq l$.\\

Очевидно, что $z(x)$ удовлетворяет граничным условиям. По условию теоремы однородная краевая задача имеет только тривиальное решение на отрезке $[0, l]$ $\implies z(x) \equiv 0 \implies G(x, \xi) = \widehat{G}(x, \xi). \quad \blacksquare$\\

\textbf{Теорема.} Если однородная краевая задача имеет только тривиальное решение, то решение краевой задачи \eqref{eq13.1} существует, единственно и выражается через функцию Грина $$u(x) = \int\limits_0^l G(x, \xi) f(\xi) d\xi, \quad 0 \leq x \leq l.$$

\textbf{Доказательство.} Покажем, что функция $u(x)$, определяемая формулой выше, является решением задачи \eqref{eq13.1}.

$u(x) = \int\limits_0^x G(x, \xi) f(\xi) d\xi + \int\limits_x^l G(x, \xi) f(\xi) d\xi = \left \{ g_0 = \frac{1}{W(\xi) p(\xi)} \right \} = \frac{u_2(x)}{g_0} \int\limits_0^x u_1(\xi) f(\xi) d\xi + \frac{u_1(x)}{g_0} \int\limits_x^l u_2(\xi) f(\xi) d\xi \quad$ (использовали \eqref{eq13.2}).

Дифференцируем: $u'(x) = \frac{u_2'(x)}{g_0} \int\limits_0^x u_1(\xi) f(\xi) d\xi + \frac{u_1'(x)}{g_0} \int\limits_x^l u_2(\xi) f(\xi) d\xi$.\\[2mm]

Вычислим: $\frac{d}{dx} \left ( p(x) \frac{du}{dx} \right ) = \frac{(u_1(x) u_2'(x) - u_2(x) u_1'(x)) p(x)}{g_0} f(x) +$ \\$+ \frac{1}{g_0} \frac{d}{dx} \left ( p(x) \frac{du_2}{dx} \right ) \cdot \int\limits_0^x u_1(\xi) f(\xi) d\xi + \frac{1}{g_0} \frac{d}{dx} \left ( p(x) \frac{du_1}{dx} \right ) \cdot \int\limits_x^l u_2(\xi) f(\xi) d\xi$.

Т.к. $Lu_1 = Lu_2 = 0$, а $(u_1 u_2' - u_2 u_1') p(x) = g_0$, то $Lu = \frac{d}{dx} \left ( p(x) \frac{du}{dx} \right ) - q(x) u(x) = f(x) + \frac{Lu_2}{g_0} \int\limits_0^x u_1(\xi) f(\xi) d\xi + \frac{Lu_1}{g_0} \int\limits_x^l u_2(\xi) f(\xi) d\xi = f(x).$\\

Убедимся в выполнении граничных условий:\\ 

$\alpha_1 u'(0) + \beta_1 u(0) = \{ \text{формулы для $u(x), u'(x)$ в начале док-ва} \} =$\\$= \frac{\alpha_1 u_1'(0)}{g_0} \int\limits_0^l u_2(\xi) f(\xi) d\xi + \frac{\beta_1 u_1(0)}{g_0} \int\limits_0^l u_2(\xi) f(\xi) d\xi =$\\$= \frac{\alpha_1 u_1'(0) + \beta_1 u_1(0)}{g_0} \int\limits_0^l u_2(\xi) f(\xi) d\xi =$\\$= \{ \alpha_1 u_1'(0) + \beta_1 u_1(0) =  0 \text{ по определению} \} = 0.$ \\

Аналогично второе условие.

Докажем единственность. Пусть $u(x),\; \widehat{u}(x)$ -- два решения задачи \eqref{eq13.1}. Тогда $z(x) = u(x) - \widehat{u}(x)$ -- решение однородной задачи, т.е. тривиальное (по условию теоремы) $\implies z(x) \equiv 0 \implies u(x) \equiv \widehat{u}(x). \quad \blacksquare$

% -------- source --------
\bigbreak
[\cite{denisov2}, пункт 3.2]
