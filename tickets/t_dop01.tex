\setcounter{section}{4}
\setcounter{subsection}{1}
\setcounter{equation}{0}
\textbf{\LARGE dop 1. Необходимые условия экстpемума функции нескольких пеpеменных. Достаточные условия.}

\rule{275pt}{0.5pt}
\begin{center}
    \textbf{Вспомогательная теория}
\end{center}

\textbf{Вторая теорема Вейерштрасса:} Если функция $f(x)$ непрерывна на некотором замкнутом и ограниченном множестве, то она достигает на этом множестве своих точных верхней и нижней граней.

\textbf{Теорема Ферма:} Если $f(x)$ дифференцируема в точке $x=c$ и имеет в этой точке локальный экстремум, то $f'(c) = 0$.  

\textbf{Опр:} Квадратичная форма $\Phi(h_1, h_2, \ldots, h_n) = \sum_{i=1}^n \sum_{j=1}^n a_{ij} h_i h_j$ с матрицей $A = \{a_{ij}\}$ называется:
\begin{itemize}
    \item \textbf{положительно (отрицательно) определенной}, если для $\forall \; h_1, \ldots, h_n$, одновременно не равных нулю, эта форма принимает строго положительные (отрицательные) значения;
    \item \textbf{знакоопределенной}, если она является либо положительно, либо отрицательно определенной;
    \item \textbf{знакопеременной}, если она принимает как строго положительные, так и строго отрицательные значения;
    \item \textbf{квазиопределенной}, если она принимает либо только неотрицательные, либо только неположительные значения, но при этом обращается в нуль для некоторых $h_1, \ldots, h_n$, одновременно не равных нулю.
\end{itemize}

\textbf{Критерий Сильвестра знакоопределенности квадратичной формы:}
\begin{enumerate}
    \item Квадратичная форма $\Phi$ с симметричной матрицей $A$ является \textbf{положительно} определенной $\Leftrightarrow$ все главные миноры матрицы $A$ положительны;
    \item Квадратичная форма $\Phi$ с симметричной матрицей $A$ является \textbf{отрицательно} определенной $\Leftrightarrow$ знаки главных миноров матрицы $A$ чередуются, причем $A_1 = a_{11} < 0$.
\end{enumerate}
\rule{275pt}{0.5pt} \\

\fbox{\textbf{Опр. 1:}} Точка $x_0 = (x_{01}, \ldots, x_{0n})$, внутренняя для области определения функции $f(x) = f(x_1, \ldots, x_n)$, называется точкой \textbf{локального экстремума} функции $f$, если для любой точки $x$ из некоторой ее окрестности $U = U(x_0)$ разность $\Delta f = f(x) - f(x_0)$ отлична от нуля и сохраняет знак. В частности, если $\Delta f> 0$, то это точка \textbf{локального минимума}, если $\Delta f < 0$, то это точка \textbf{локального максимума}.\\

\textbf{Теорема 1 \textit{(необходимое условие)}:} Если у функции $f(x)$ в точке $x_0$ существуют все частные производные $f'_{x_k}$, и эта точка является точкой локального экстремума, то все частные производные в ней равны нулю, то есть $f'_{x_k}(x_0) = 0, \; k = 1, \ldots, n$.

$\blacktriangleright\;$ Зафиксируем у функции $f(x)$ все переменные, кроме $x_k \;(1\leq k\leq n)$, и рассмотрим функцию $\varphi(x_k) = f(x_{01}, \ldots, x_{0 k-1}, x_k, x_{0 k+1}, \ldots, x_{0 n})$ -- функцию одной переменной. Очевидно, что для функции $\varphi(x_k)$ точка $x_{0k}$ является также точкой локального экстремума. По теореме Ферма имеем, что $\varphi'(x_{0k}) = 0 \;\implies\; f'_{x_k} (x_0) = 0, \; 1\leq k \leq n$.\;$\blacksquare$\\

\fbox{\textbf{Опр. 2:}} Точка $x_0 = (x_{01}, \ldots, x_{0n})$ называется \textbf{стационарной} для функции $f(x) = f(x_1, \ldots, x_n)$, если она внутренняя для ее области определения, и все частные производные в ней определены и равны нулю: $f'_{x_k}(x_0) = 0, \; k = 1, \ldots, n$.\\

\textbf{Теорема 2 \textit{(достаточные условия)}:} Пусть функция $f(x) = f(x_1, \ldots, x_n)$ от $n$ независимых переменных один раз дифференцируема в некоторой окрестности точки $x_0 = (x_{01}, \ldots, x_{0n})$, и дважды дифференцируема в самой точке $x_0$. Пусть $x_0$ -- стационарная точка, то есть дифференциал функции $f$ равен нулю в точке $x_0$ ($d f(x_0) = 0$). Тогда если второй дифференциал $d^2 f(x_0)$ представляет собой знакоопределенную квадратичную форму, то $x_0$ -- точка локального экстремума. При этом, если форма $d^2f(x_0)$ положительно (отрицательно) определена, то $x_0$ -- точка локального миниммума (максимума). Если же квадратичная форма $d^2 f(x_0)$ знакопеременна, то локального экстремума в точке $x_0$ нет.

$\blacktriangleright\;$ Разложим разность $\Delta f$ по формуле Тейлора при $n = 2$ с остаточным членом в форме Пеано:
$$
\Delta f = f(x) - f(x_0) = d f(x_0) + \frac{d^2 f(x_0)}{2!} + \overline{o}(\|\Delta x\|^2) =$$
$$= \{ d f(x_0) = 0\} = \frac{1}{2!}\,\sum_{i = 1}^n \sum_{j =1}^n f''_{x_i x_j}(x_0) \Delta x_i \Delta x_j + \overline{o}(\|\Delta x\|^2).
$$
Обозначим $h_i = \frac{\Delta x_i}{\|\Delta x\|}, \; a_{ij} = f''_{x_i x_j}(x_0), \; \alpha = \alpha (\|x\|) = \frac{\overline{o}(\|\Delta x\|^2)}{\|\Delta x\|^2}$. 
Отметим, что $\alpha(\|x\|)$ -- бесконечно малая при $\|\Delta x\| \to 0$. 
Тогда $\Delta f$ представляется в виде $\Delta f = \frac{1}{2}\|\Delta x\|^2\, \left( \sum_{1\leq i, j \leq n} a_{ij} h_i h_j + \alpha\right)$. 
Заметим, что вектор $h = \{ h_1, \ldots, h_n\}$ имеет норму $\|h\| = \sqrt{h_1^2 + \ldots + h_n^2} = 1$, то есть это элемент единичной сферы $S^{n-1}$ пространства $\mathbb{R}^n$.

\textbf{1)} Рассмотрим случай, когда $d^2 f(x_0)$ -- \textbf{положительно определенная} квадратичная форма. 
В этом случае квадратичная форма $\Phi(h) = \sum_{1\leq , j \leq n} a_{ij}h_i h_j$ -- также положительно определенная непрерывная функция на единичной сфере $S^{n-1}$. 
Сфера $S^{n-1}$ является замкнутым ограниченным множеством в $\mathbb{R}^n$, следовательно, по второй теореме Вейерштрасса функция $\Phi(h)$ достигает на $S^{n-1}$ своего инфимума, то есть существует такая точка $\xi = (\xi_1, \ldots, \xi_n) \in S^{n-1}$, в которой $\Phi(\xi) = \inf_{h \in S^{n-1}} \Phi(h) = \mu$. 
Поскольку всюду на сфере $\Phi(h) > 0$, то $\Phi(h) \leq \Phi(\xi) = \mu > 0$. Воспользовавшись этим, получаем 
$$
\Delta f = \frac{1}{2} \|\Delta x\|^2 (\Phi(h) + \alpha) \geq \frac{1}{2} \|\Delta x\|^2 (\mu + \alpha).
$$
Так как $\alpha \to 0$ при $\|\Delta x\| \to 0$, то $\exists \, \delta = \delta (\mu) > 0 \,:\; \|\Delta x\| < \delta \implies |\alpha| < \frac{\mu}{2}$.
При этих условиях 
$$
\Delta f \geq \frac{1}{2} \|\Delta x\|^2 (\mu + \alpha) > \frac{1}{2} \|\Delta x\|^2 (\mu - \frac{\mu}{2}) = \frac{1}{2} \|\Delta x\|^2 \frac{\mu}{2} > 0.
$$
Итак $\forall\, x \in \mathbb{R}^n : \; \|\Delta x\| < \delta \implies \Delta f > 0 $. Следовательно, точка $x_0$ -- точка локального минимума по определению 1.

\textbf{2)} В случае \textbf{отрицательно определенной} квадратичной форму $d^2 f(x_0)$ совершенно аналогично доказывается, что $x_0$ -- точка локального максимума.

\textbf{3)} Пусть теперь $d^2 f(x_0)$ -- \textbf{знакопеременная} квадратичная форма. Тогда имеем 
$$
\Delta f = \frac{1}{2} \rho^2 (\Phi(h) + \alpha),\quad \rho = \|\Delta x\|,
$$
где $\Phi(h)$ -- знакопеременная квадратичная форма на единичной сфере $S^{n-1}$. 
Следовательно $ \exists \; h', \; h'' \in S^{n-1} \,: \; \Phi(h') <0, \; \Phi(h'') > 0$. 
При этом заметим, что $\alpha = \alpha (\rho)$ и $\alpha \to 0$ при $\rho \to 0$, а $\Phi(h)$ от $\rho$ не зависит.
Поэтому взяв $\rho_0$ достаточно малым, можно добиться, чтобы $|\alpha| = |\alpha(\rho_0)| < \min \big\{|\Phi(h')| / 2, \;  |\Phi(h'')| / 2\big\}$. 
При этих условиях будет одновременно $ \Phi(h') + \alpha(\rho_0) < 0, \; \Phi(h'') + \alpha(\rho_0) > 0$. 
Тогда для точек $x' = x_0 + \rho_0 h', \; x'' = x_0 + \rho_0 h''$ будем иметь
$$
(\Delta f)_1 = f(x') - f(x_0) = \frac{1}{2} \rho_0^2 \big(\Phi(h') + \alpha(\rho_0) \big) < 0,
$$
$$
(\Delta f)_2 = f(x'') - f(x_0) = \frac{1}{2} \rho_0^2 \big(\Phi(h'') + \alpha(\rho_0) \big) > 0.
$$
Итак, приращение функции меняет знак $\implies$ точка $x_0$ не является точкой экстремума функции $f(x)$. $\;\blacksquare$\\

\textbf{Замечание:} В случаях \textbf{квази-определенности} (полуопределенности) квадратичной формы второго дифференциала $d^2 f(x_0)$ ответ о существовании локального экстремума в точке $x_0$ дать нельзя.


% -------- source --------
\bigbreak
[\cite[page 69-74]{Sadovn_func_mhogih}]
