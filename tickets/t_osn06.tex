\setcounter{section}{15}
\setcounter{subsection}{6}
\setcounter{equation}{0}
\textbf{\LARGE osn 6. Криволинейный интеграл, формула Грина.}

$\mathLet ~ \varphi(t), \psi(t)$ непр. на $[\alpha,\beta]$. 
Если рассматривать $t$ как время, эти функции определяют закон движения по плоскости точки $M$ с координатами 
$x = \varphi(t), y = \psi(t), ~ \alpha < t < \beta$. Множество $\{M\}$ всех точек $M$, координаты $x,y$ которых определяются уравнениями $\varphi(t), \psi(t)$, называется \textbf{простой плоской кривой} $L$, если различным значениям параметра $t$ из $[\alpha, \beta]$ отвечают различные точки этого множества.

% $\mathLet ~ \varphi(t), \psi(t) \in C[\{t\}]$. Уравнения $x = \varphi(t), y = \psi(t)$ задают параметрически кривую L, если $\exists$ такая система сегментов $\{[t_{i-1}, t_i]\}$, разбивающих множество $\{t\}$, что для значений $t$ из каждого данного сегмента этой системы все уравнения определяют простую кривую.

\textbf{Спрямляемая кривая} --- кривая, имеющая конечную длину.

\textbf{Длина кривой} -- это предел последовательности длин ломаных, вписанных в эту линию, при условии, что длина наибольшего звена $\rightarrow 0$.\\

$\mathLet ~ x = \varphi(t), y = \psi(t) \in C[\alpha, \beta]$. Тогда кривая $L$, определяемая $x, y$, спрямляема и длину $l$ ее дуги можно вычислить по формуле
$$l = \int^{\beta}_{\alpha} \sqrt{\varphi'^2(t) + \psi'^2(t)}dt$$

\faEye \ произвольную спрямляемую кривую $L$ на плоскости $Oxy$, не имеющую точек самопересечения и самоналегания, незамкнутую, ограниченную точками $A,~B$, описывающуюся параметрическими ур-ями:
$$\begin{cases} x=\varphi(t)&\\ y=\psi(t)\end{cases},~t\in [a,b], A=(\varphi(a),\psi(a)), B=(\varphi(b),\psi(b))$$

\mathLet \ на кривой L определены три непрерывные вдоль этой кривой функции $f(x,y)=f(M),~P(x,y)=P(M),~Q(x,y)=Q(M)$.

\faEye \ разбиение отрезка 
$[a,b]:~a=t_0 < t_1 < \dots < t_n = b,~$
$\Delta t_k = t_k-t_{k-1}, ~ M_k = M_k(\varphi(t_k),\psi(t_k))$.

$\Delta l_k = |\smile M_{k-1}M_k|$ (длина дуги), $\Delta \equiv \displaystyle\max_{1\leqslant k\leqslant n} \Delta l_k$.

Выберем точки $N_k(\xi_k, \eta_k) \in \smile M_{k-1}M_k,~\xi_k=\varphi(\tau_k),~\eta_k=\psi(\tau_k),~\tau_k\in [t_{k-1},t_k]$.

$\Delta x_k = x_k - x_{k-1},~x_k = \varphi(t_k),~\Delta y_k = y_k - y_{k-1},~y_k = \psi(t_k)$

\faEye \ три интегральные суммы:\\
1. $\sigma_1=\displaystyle\sum_{k=1}^n f(N_k)\Delta l_k$\;  2. $\sigma_2=\displaystyle\sum_{k=1}^n P(N_k)\Delta x_k$  \;3. $\sigma_3=\displaystyle\sum_{k=1}^n Q(N_k)\Delta y_k$

Число $I_s,~s=1,2,3$ называется \textbf{пределом интегральной суммы} $\sigma_s$ при $\Delta \rightarrow 0$, если $\forall\varepsilon>0~\exists\delta>0:~\Delta<\delta\implies|I_s-\sigma_s|<\varepsilon$ независимо от выбора точек $N_k\in\smile M_{k-1}M_k$.

Если существует предел $I_1$ интегральной суммы $\sigma_1$ при $\Delta \rightarrow 0$, то он называется \textbf{криволинейным интегралом 1 рода} от функции $f$ по кривой $L$.
$$I_1=\displaystyle\lim_{\Delta\to 0}\sigma_1
= \int\limits_{L}f(x,y)dl 
%=\int\limits_{\smile AB}f(x,y)dl 
=\int\limits_{a}^{b} f(\varphi(t), \psi(t)) \sqrt{ \varphi_t^{'2}(t) + \psi_t^{'2}(t) } dt $$

Если существуют пределы $I_2,~I_3$ интегральных сумм $\sigma_2,~\sigma_3$ при $\Delta \rightarrow 0$, то они называются \textbf{криволинейными интегралами 2} рода от функций $P,~Q$ по кривой $AB$.
$$I_2=\displaystyle\lim_{\Delta\to 0}\sigma_2= \int\limits_{\smile AB}P(x,y)dx  = \int\limits_a^b P(\varphi(t), \psi(t)) \varphi'(t) \,dt$$
$$I_3=\displaystyle\lim_{\Delta\to 0}\sigma_3= \int\limits_{\smile AB}Q(x,y)dy =\int\limits_a^b Q(\varphi(t), \psi(t)) \varphi'(t) \,dt$$

$I_2+I_3=\int\limits_{\smile AB}P(x,y)dx+\int\limits_{\smile AB}Q(x,y)dy = \int\limits_{\smile AB}P(x,y)dx+Q(x,y)dy$ 

-- \textbf{общий криволинейный интеграл 2 рода}.

% Из определения криволинейных интегралов следует, что:
% \begin{enumerate}
%     \item криволинейный интеграл первого рода не зависит от того, в каком направлении пробегает кривая $L$, а для криволинейного интеграла второго рода изменение направления кривой ведёт к изменению знака, т.е. $\int\limits_{\smile AB}P(x,y)dx+Q(x,y)dy=-\int\limits_{\smile BA}P(x,y)dx+Q(x,y)dy$
%     \item физически криволинейный интеграл первого рода представляет собой массу кривой $L$, линейная плотность которой равна $f(x, y)$;

%     общий линейный интеграл второго рода физически представляет собой работу по перемещению материальной точки $A$ в точку $B$ вдоль кривой $L$ под действием силы, имеющей составляющие $P(x, y)$ и $Q(x, y)$.
% \end{enumerate}

% \todo{Во второй части билета какая-то херня. Закоментированны} \\
% \todo{совершенно рандомные куски ибо не влезает. Надо что-то придумать.}
\begin{center}
    \textbf{Формула Грина}
\end{center}
% Область $D$ называется\textbf{ односвязной}, если любая кусочно гладкая замкнутая без самопересечения кривая, расположенная в $D$, ограничивает область, все точки которой также принадлежат $D$.

$\mathLet ~ \pi$ --- плоскость в пространстве $E_3$, $\overrightarrow{k}$ --- единичный вектор нормали к $\pi$, $D$ --- односвязная область на $\pi$. \mathLet \ далее, область $D$ удовлетворяет следующим условиям:
\begin{enumerate}
    \item граница $C$ области $D$ является замкнутой кусочно-гладкой кривой без особых точек;
    \item на плоскости $\pi$ можно выбрать такую прямоугольную декартову систему координат, что все прямые, параллельные координатным осям, пересекают $C$ не более чем в двух точках.
\end{enumerate}

% $\mathLet ~ \overrightarrow{t}$ --- единичный вектор касательной к кривой $C$, согласованной с $\overrightarrow{k}$.
% т.е. положительное направление обхода кривой $C$ совпадает в точке приложения $\overrightarrow{t}$ с направлением этого вектора, и если смотреть с конца нормали $\overrightarrow{k}$, то контур $C$ ориентирован положительно. 
% (Его обход осуществляется против часовой стрелки). Говорят, что ориентация кривой $C$ согласована с нормалью <<по правилу штопора>>.

\textbf{Опр.} \textbf{Векторным полем} в $\mathcal{R}^3$ называется векторная функция, определенная в $\mathcal{R}^3$.

\textbf{Опр.} \textbf{Ротором} векторного поля $p$ называется $rot~p = \nabla \times p$. 
% В ОНБ имеет вид:

% $\left(\frac{\partial a_z}{\partial y} - \frac{\partial a_y}{\partial z}\right) \overline{e}_1$ + $\left(\frac{\partial a_x}{\partial z} - \frac{\partial a_z}{\partial x}\right) \overline{e}_2$ + $\left(\frac{\partial a_y}{\partial x} - \frac{\partial a_x}{\partial y}\right) \overline{e}_3$

\textbf{Формула Грина.} Пусть область $D$ удовлетворяет условиям 1 и 2, $\overrightarrow{t}$ – единичный вектор касательной к кривой C,
согласованный с вектором k (т.е. положительное направление
обхода C совпадает с направлением касательной), векторное
поле $\overrightarrow{p}(\overrightarrow{r})$ определено и непрерывно в $\overline{D} = D \cup C$, дифференцируемо в $D$ и производная по любому направлению
непрерывна в $D$. Тогда
\begin{equation}\label{eq06.1}
    \iint\limits_{\overline{D}}(\overrightarrow{k},rot~\overrightarrow{p})d\sigma = \oint\limits_{C} (\overrightarrow{p},\overrightarrow{t}) dl 
\end{equation}
% Выражение справа обычно называют \textbf{циркуляцией векторного поля} $a$ по кривой $C$, а выражение слева --- \textbf{потоком векторного поля} $rot~a$ через область $D$.

% \includegraphics[scale=0.06]{pics/osn06_L.jpeg}

$\blacktriangleright\;$
Введем ортонорм. декартову сис. коорд.: $Oz \uparrow\uparrow \overrightarrow{k}$, $O_{xy}$ -- в плоскости $\pi$ и выбраны так, чтобы выполнялось условие 2 на $D$.

Запишем координаты используемых векторов в введенной системе координат: $\overrightarrow{k} = \{0, 0, 1\}, \; \overrightarrow{p}(x,y) = \{P(x,y), Q(x,y), R(x,y)\},\; \overrightarrow{t} = \{\cos(\alpha), \cos(\beta), 0\}$.

Тогда $(\overrightarrow{k},rot~\overrightarrow{p}) = \frac{\partial Q}{\partial x} - \frac{\partial P}{\partial y}, \;\; d\sigma = dx\,dy$, так как множество $D$ находится в плоскости $z(x,y) = 0$.

Подставив полученные соотношения в \eqref{eq06.1}, получим \textbf{формулу Грина в ОНБ}
\begin{equation}\label{eq06.2}
    \iint\limits_{\overline{D}} (\frac{\partial Q}{\partial x} - \frac{\partial P}{\partial y}) dx dy = \oint\limits_{C} (P \cos(\alpha) + Q \cos(\beta)) dl = \oint\limits_{C} (P dx + Q dy).
\end{equation} 

Формулу \eqref{eq06.2} достаточно доказать <<покомпонентно>>: 
$$ I_1 = -\iint\limits_{\overline{D}}  \frac{\partial P}{\partial y} dx dy = \oint\limits_{C} P dx, \;\; I_2=\iint\limits_{\overline{D}}  \frac{\partial Q}{\partial x} dx dy = \oint\limits_{C} Q dy$$
Докажем для $I_1$ (для $I_2$ аналогично). 

По условию 2 на область $D$ прямая, параллельная оси $Oy$ пересекает контур $C$ не более, чем в двух точках: $y_1(x)$ и $y_2(x)$ ($y_1(x) \leq y_2(x)$). Пусть $[x_1, x_2]$ -- проекция множества $\overline{D}$ на ось $Ox$. Разбиваем кривую $C = C_1 \cup C_2, \; C_1: y = y_1(x), \; C_2: y = y_2(x)$. Учитывая, что движение по $C_1$ происходит в положительном направлении, а по $C_2$ -- в отрицательном, имеем:\\
$
I_1 = - \int\limits_{x_1}^{x_2} dx \int\limits_{y_1(x)}^{y_2(x)} \frac{\partial P}{\partial y}dy = \int\limits_{x_1}^{x_2} P(x, y_1(x)) dx - \int\limits_{x_1}^{x_2}P(x, y_2(x)) dx = \int\limits_{C_1} P(x,y) dx - (-\int\limits_{C_2} P(x,y) dx) = \oint\limits_{C} P dx. \; \blacksquare$\\

Теперь обоснуем последнее равенство в формуле \eqref{eq06.2}. Рассмотрим гладкую кривую $C = \smile AB$ и выберем в качестве параметра длину дуги $t = |\smile AM|, \; M \in C$. Представим кривую $C$ уравнениями $x = x(t), \; y = y(t), \; 0 \leq t \leq L, \; L = |\smile AB|$. 

Функции $x(t), \; y(t)$ имеют непрерывные производные $x'(t), \; y'(t)$. Обозначим через $\alpha$ угол, составленный с осью $Ox$ касательной $\overrightarrow{a}$ к кривой $C$, направленной в сторону роста $t$. Пусть $\beta$ -- дополнение угла $\alpha$ до $\pi / 2$. Имеет место следующее утверждение.

\textbf{Лемма.} $\cos(\alpha) = x'(t), \; \sin(\alpha) = \cos(\beta) = y'(t), \; t \in [0, L].$

$\blacktriangleright\;$ Пусть точке $M$ отвечает значение $t$ длины дуги. Придадим приращение $\Delta t > 0$, тогда $t + \Delta t$ определяет точку $M_1 \in C$, лежащую от $M$ в сторону возрастания дуг. 

Обозначим через $\alpha'$ угол, который составляет секущая $\overrightarrow{M M_1}$ и ось $Ox$. Приращению $\Delta t$ отвечают приращения $\Delta x = |\overrightarrow{M M_1}| \cos(\alpha'), \; \Delta y = \overrightarrow{M M_1} \sin(\alpha')$. Так как $\Delta t = |\smile MM_1|$, то 
$$
\cos(\alpha') = \frac{\Delta x}{\Delta t} \cdot \frac{|\smile MM_1|}{|\overrightarrow{M M_1}|}, \;\; \sin(\alpha') = \frac{\Delta y}{\Delta t} \cdot \frac{|\smile MM_1|}{|\overrightarrow{M M_1}|}.
$$
Но $\frac{|\smile MM_1|}{|\overrightarrow{M M_1}|} \to 1$ при $\Delta t \to 0$ $\implies$
$\cos(\alpha) = \frac{dx}{dt}, \;\; \sin(\alpha) = \cos(\beta) = \frac{dy}{dt}.
\; \blacksquare$

Продолжим обоснование перехода в \eqref{eq06.2}. Пусть вдоль кривой $C$ задана непрерывная функция $f(M) = f(x,y)$. Тогда $\int\limits_C f(M) dx = \int\limits_0^L f(x(t), y(t)) x'(t) dt = \int\limits_0^L f(x(t), y(t)) \cos(\alpha) dt = \int\limits_C f(M) \cos(\alpha) dl$, где мы учли, что $dl = \sqrt{x'^2(t) + y'^2(t)} dt = \sqrt{\cos^2(\alpha) + \sin^2(\alpha)}dt = dt$. 
% Аналогично $\int\limits_C f(M) dy = \int\limits_C f(M) \cos(\beta) dl$. 

Таким образом, если функции $P,\; Q$ непрерывны на $C$, то 
$$
\int\limits_C P dx + Q dy = \int\limits_C (P \cos(\alpha) + Q \cos(\beta)) dl.
$$
Эта формула и использовалась в последнем переходе в \eqref{eq06.2}.

% Таким образом значение функции по третьей координате ($k$) не влияет на значение.
% Поэтому, не теряя общности, можем рассматривать функции, определенные только на плоскости: $a=\{P(x,y),Q(x,y)\}$.
% Докажем для этого прредставления для простой области.
% Пусть $L$ -- положительно ориентированная кусочно-гладкая замкнутая кривая на плоскости, а $D$ -- простая плоская область, ограниченная кривой $L$.
% Функции $P=P(x,y)$, $Q=Q(x,y)$ определены в области $D$ и имеют непрерывные частные производные $\frac{\partial P}{\partial y}, \frac{\partial Q}{\partial x}$.

% \begin{proof}
% Докажем по отдельности:

% \textbf{(1)}$ \oint\limits_{L}Pdx = \iint\limits_{\mathcal{D}}(-\frac{\partial P}{\partial y})dxdy$, 
% \textbf{(2)}$ \oint\limits_{L}Qdx = \iint\limits_{\mathcal{D}}(-\frac{\partial Q}{\partial y})dxdy$.

% $D: x_1 \leq x \leq x_2, ~ y_1(x) \leq y \leq y_2(x)$

% $(1) =\iint\limits_{\mathcal{D}}\frac{\partial P}{\partial y}dxdy = $
% $ -\int_{x_1}^{x_2}dx\int_{y_1(x)}^{y_2(x)} \left(\frac{\partial P}{\partial y} \right)dy = $
% $ -\int_{x_1}^{x_2}(P(x,y_2(x)) - P(x, y_1(x)))dx = -\int_{x_1}^{x_2}P(x, y_2(x))dx + \int_{x_1}^{x_2}P(x, y_1(x))dx = \int\limits_{\smile BA}P(x,y)dx + \int\limits_{\frown AB}P(x,y)dx$

% Аналогично, спроецируем D на $O_y$ - докажем вторую формулу.
% \end{proof}

% \textbf{Формулировка}
% Пусть $C$ -- положительно ориентированная кусочно-гладкая замкнутая кривая на плоскости, а $D$ -- область, ограниченная кривой $C$. 
% Если 
% %функция $P = P(x,y)$, $Q = Q(x,y)$ определены в области $D$ и имеют непрерывные частные производные
% $\frac{\partial P}{\partial y}$, $\frac{\partial Q}{\partial x} \in \mathcal{C}(D)$, то

% $\oint\limits_{C} P \,dx + Q \,dy = \iint\limits_{D} \left( \frac{\partial Q}{\partial x} - \frac{\partial P}{\partial y} \right) \,dx\,dy$

% На символе интеграла часто рисуют окружность, чтобы подчеркнуть, что кривая $C$ замкнута.

% \begin{proof}
% \textbf{Доказательство ф. Грина для простой области}
% %$D$ -- область, правильная в направлении $OY$, ограниченная замкнутой кривой $C$
% \mathLet \ область $D$ -- криволинейная трапеция (область, правильная в направлении $OY$)
% : $D = \{ (x,y)|a \le x \le b, y_1(x) \le y \le y_2(x) \}$

% Для кривой $C$, ограничивающей область $D$ зададим направление обхода по часовой стрелке. Тогда:

% $\iint\limits_{D} \frac{\partial P}{\partial y} \,dx\,dy = \int\limits_{a}^{b}dx \int\limits_{y_1(x)}^{y_2(x)} \frac{\partial P}{\partial y} \,dy = \int\limits_{a}^{b} (P(x,y_2(x)) - P(x,y_1(x))) \,dx =$
% $= \int\limits_{a}^{b} P(x,y_2(x)) \,dx - \int\limits_{a}^{b} P(x,y_1(x)) \,dx \quad (1)$

% Заметим, что оба полученных интеграла можно заменить криволинейными интегралами:
% $\int\limits_{C_1} P(x,y) \,dx = -\int\limits_{-C_1} P(x,y) \,dx = -\int\limits_{a}^{b} P(x,y_1(x)) \,dx \quad (2)$
% $\int\limits_{C_3} P(x,y) \,dx = \int\limits_{a}^{b} P(x,y_2(x)) \,dx \quad (3)$
% Интеграл по $C_1$ берётся со знаком «минус», так как согласно ориентации контура $C$ направление обхода данной части -- от $b$ до $a$.

% Криволинейные интегралы по $C_2$ и $C_4$ будут равны нулю, так как $x = \operatorname{const}$:
% $\int\limits_{C_2} P(x,y) \,dx = 0 \quad (4)$
% $\int\limits_{C_4} P(x,y) \,dx = 0 \quad (5)$

% Заменим в (1) интегралы согласно (2) и (3), а также прибавим (4) и (5), равные нулю и поэтому не влияющие на значение выражения:\\
% $\iint\limits_{D} \frac{\partial P}{\partial y} \,dx\,dy = \int\limits_{C_1} P(x,y) \,dx + \int\limits_{C_3} P(x,y) \,dx + \int\limits_{C_2} P(x,y) \,dx + \int\limits_{C_4} P(x,y) \,dx$

% Так как обход по часовой стрелке при правой ориентации плоскости является отрицательным направлением, то сумма интегралов в правой части является криволинейным интегралом по замкнутой кривой $C$ в отрицательном направлении:
% $\iint\limits_{D} \frac{\partial P}{\partial y} \,dx\,dy = -\int\limits_{C} P(x,y) \,dx \quad (6)$

% Аналогично доказывается формула:\\
% $\iint\limits_{D} \frac{\partial Q}{\partial x} \,dx\,dy = \int\limits_{C} Q(x,y) \,dy \quad (7)$
% если в качестве области $D$ взять область, правильную в направлении $OX$.

% Сложим (6) и (7):
% $\int\limits_{C} P \,dx + Q \,dy = \iint\limits_{D} \left( \frac{\partial Q}{\partial x} - \frac{\partial P}{\partial y} \right) \,dx\,dy$
% \end{proof}

% -------- source --------
\bigbreak
[\cite[page 463-472]{lomov}]
