\setcounter{section}{4}
\setcounter{subsection}{5}
\setcounter{equation}{0}
\textbf{\LARGE dop.5 Ряд Лоpана. Классификация изолиpованных особых точек.}

\textbf{Определение 1.}
\textbf{Рядом Лорана} называется функциональный ряд вида: $\sum_{n=-\infty}^{n=\infty}c_n(z-z_0)^n\textbf{(1)} = \sum_{n=0}^{n=\infty}c_n(z-z_0)^n(\text{ряд.1}) +  \sum_{n=1}^{n=\infty}\frac{c_{-n}}{(z-z_0)^n}(\text{ряд.2})$, где $z$ переменная $\in \mathbb{C}\backslash\{z_0\}$, а $c_n$ коэффициенты $\in \mathbb{C}$.

\noindent Говорят, что ряд (1) сходится в т. $z$, если в ней сходятся ряд.1 и ряд.2.

    1) Ряд.1 называют \textbf{правильной} частью. Если его радиус сходимости $R_1=0$, то он сходится лишь в т.$z_0$, а ряд (1) не сходится нигде. Если $R_1>0$, то в круге $|z-z_0|<R_1$ ряд.1 сходится абсолютно к некоторой функции $f_1(z)$.
    
    2) Ряд.2 -- называют \textbf{главной} частью, он не является степ рядом, но приводится к нему заменой $\rho = \frac{1}{z-z_0}$. Если радиус сходимости ряда $\sum_{n=1}^{n=\infty}c_{-n}\rho^n$\textbf{(2)} равен 0, то и ряд (1) и ряд.2 не сходятся. Если радиус сходимости (2) $R^{-1}_2 > 0$, то ряд (2) сходится абсолютно в круге $|\rho|<R^{-1}_2$ $\Rightarrow$ ряд.2 сходится абсолютно в области $|z-z_0|>R_2$  к некоторой функции $f_2(z)$. Если $R_1<R_2$, то области сход. рядов не пересекаются и ряд Лорана не сходится нигде. Если $R_1=R_2=R$, то общие точки сходимости могут лежать лишь на $|z-z_0|=R$ и их наличие требует отдельного исследования. Если $R_1>R_2$, то оба ряда абсолютно сходятся в кольце $D: R_2<|z-z_0|<R_1$, ряд (1) абсолютно сходится там же к функции $f(z) = f_1(z)+f_2(z)$.
    
\textbf{Замечание:} $\sqsupset$ ряд (1) абс.cход в кольце $D$ к функции $f(z)$. Покажем, что коэффициенты этого ряда однозначно определяются его суммой $f(z)$. Рассмотрим ряд (1) в точках окружности $\phi: |z-z_0|=\rho$, где $R_2<\rho<R_1$. На этой окружности как на компакте, ряд сходится равномерно. Равномерная сходимость сохраняется при умножении каждого члена ряда на функцию ограниченную на $\phi$. Фикс $k$ и рассмотрим функцию $\frac{1}{2\pi i(z-z_0)^{k+1}} \Rightarrow \frac{f(z)}{2\pi i(z-z_0)^{k+1}} = \sum_{n=-\infty}^{n=\infty}\frac{1}{2\pi i}c_n(z-z_0)^{n-k-1} \Longleftrightarrow \frac{1}{2\pi i} \oint_\phi \frac{f(z)}{(z-z_0)^{k+1}} \,dz = \sum_{n=-\infty}^{n=\infty}\frac{1}{2\pi i}c_n\oint_\phi(z-z_0)^{n-k-1} \,dz$. Интеграл в правой части $\neq0$ только при $n-k-1=-1 \Longleftrightarrow n=k$(в лекциях Домриной считался) при этом он равен $2\pi i \Rightarrow c_k = \frac{1}{2\pi i} \oint_\phi \frac{f(z)}{(z-z_0)^{k+1}} \,dz$ определены однозначно.

\textbf{Теорема 1.}
    Ф-ция $f(z)\in A(D), D: R_2<|z-z_0|<R_1$, может быть представлена рядом Лорана по степеням $(z-z_0)$ причем это представление единственно.
    
$\blacktriangleright\;$
$\Longrightarrow$: доказано в замечании.\\
$\Longleftarrow$ : Фикс произвольную точку $z \in D$, построим вспомогательное кольцо $D'$ c тем же центром в $z_0$, $D' \subset D$ и $z \subset int(D')$. $\sqsupset$ $\text{Г}_1^{'}: |\rho - z_0|=R_1'$ и $\text{Г}_2^{'}|\rho - z_0|=R_2'$ -- внутрення и внешняя границы кольца $D'$, тогда $f(z) = \frac{1}{2\pi i} \oint_{ \text{Г}_1^{'}} \frac{f(\rho)}{\rho - z} \,d\rho - \frac{1}{2\pi i} \oint_{ \text{Г}_2^{'}} \frac{f(\rho)}{\rho - z} \,d\rho(1)$. Т.к $|\frac{z-z_0}{\rho-z_0}|<1$ для $\forall$ точек $\rho \in \text{Г}_1^{'}$, то подынтегральную дробь $\frac{1}{\rho-z}$ можно заменить $\infty$ геом.прогрессией $\frac{1}{\rho-z} =\frac{1}{\rho-z_0}\cdot\frac{1}{1-\frac{z-z_0}{\rho-z_0}} = \frac{1}{\rho-z_0}\sum_{n=0}^{\infty}\frac{(z-z_0)^n}{(\rho-z_0)^n} = \sum_{n=0}^{\infty}\frac{(z-z_0)^n}{(\rho-z_0)^{n+1}} \Longleftrightarrow \frac{f(\rho)}{\rho-z_0}=\sum_{n=0}^{\infty}\frac{f(\rho)(z-z_0)^n}{(\rho-z_0)^{n+1}}(2)$ Ряд в правой части $	\rightrightarrows$ на $\text{Г}_1^{'}$ т.к мажорируется $\max_{\rho \in \text{Г}_1^{'}}|f(\rho)|\sum_{n=0}^{\infty}\frac{(z-z_0)^n}{(\rho-z_0)^{n+1}} \Rightarrow$ можно почленно интегрировать (2) по окружности $\text{Г}_1^{'}$: $\oint_{ \text{Г}_1^{'}}\frac{f(\rho)}{\rho-z_0} \,d\rho =\sum_{n=0}^{\infty} \oint_{ \text{Г}_1^{'}} \frac{f(\rho)(z-z_0)^n}{(\rho-z_0)^{n+1}} \,d\rho \Longleftrightarrow \{ c_n = \frac{1}{2\pi i} \oint_{ \text{Г}_1^{'}} \frac{f(\rho)}{(\rho-z_0)^{n+1}} \,d\rho\ \text{, }n=0..\infty\}$ $\Longleftrightarrow$  $\frac{1}{2\pi i}\oint_{ \text{Г}_1^{'}}\frac{f(\rho)}{\rho-z_0} \,d\rho = \sum_{n=0}^{\infty}c_n(z-z_0)^n\textbf{(3)}$.

\noindentРассмотрим второй интеграл в (1). Для $\forall$ точки $\rho \in  \text{Г}_2^{'}$ выполнено $\mu=\frac{|\rho-z_0|}{|z-z_0|}<1 \Rightarrow -\frac{1}{\rho-z} = \frac{1}{z-z_0-(\rho-z_0)} = \frac{1}{z-z_0}\frac{1}{1-\frac{\rho-z_0}{z-z_0}}=\sum_{n=0}^{\infty}\frac{(\rho-z_0)^n}{(z-z_0)^{n+1}}=\sum_{n=1}^{\infty}\frac{(\rho-z_0)^{n-1}}{(z-z_0)^{n}}=
\sum_{n=1}^{\infty}\frac{(z-z_0)^{-n} }{(\rho-z_0)^{-n+1}}\textbf{(4)}$. Получается равномерно сходящийся ряд на $\text{Г}_2^{'}$ т.к мажорируется числовой прогрессией со знаменателем $\mu$. Равномерная сходимость (4) сохранится и после умножения каждого члена на ограниченную в $\text{Г}_2^{'}$ ф-цию $\frac{f(\rho)}{2\pi i}$. Интегрируя почленно $-\frac{f(\rho)}{2\pi i(\rho-z)}=\sum_{n=1}^{\infty}\frac{f(\rho)(z-z_0)^{-n} }{(\rho-z_0)^{-n+1}}$ по окружности $\text{Г}_2^{'}$ и полагая $c_{-n} = \frac{1}{2\pi i} \oint_{ \text{Г}_1^{'}} \frac{f(\rho)}{(\rho-z_0)^{-n+1}} \,d\rho \text{, }n=1..\infty $. Имеем $\frac{1}{2\pi i} \oint_{ \text{Г}_2^{'}} \frac{f(\rho)}{\rho-z} \,d\rho = \sum_{n=1}^{\infty} c^{-n}(z-z_0)^{-n}\textbf{(5)}$. Заменяя оба интеграла в (1) на их разложения (3) и (5) приходим к ряду Лорана.$\blacksquare$

\rule{275pt}{0.5pt} \\
\textbf{ \large Классификация изолиpованных особых точек}

$\sqsupset$ $D:0<|z-z_0|<R$-проколотая окрестность точки $z_0\neq\infty$ и $f(z) \in A(D)$. Точка $z_0$ для функции $f(z)$ является \textbf{изолированной особой точкой}. $D$ можно рассматривать как кольцо с центром в т.$z_0$ и внутренним радиусом 0. По теореме Лорана $f(z)$ может быть разложена в $D$ в ряд Лорана (1) по степеням $z-z_0$.

Для этого ряда имеются 4 возможности:

    1)Точка $z_0$ – \textbf{устранимая особая точка} $f(z)$, если главная часть ряда Лорана (1) равна нулю.
    
    2)Точка $z_0$ – \textbf{полюс} $f(z)$, если главная часть ряда Лорана (1) содержит конечное число членов.   
    
    3)Точка $z_0$ – \textbf{полюс порядка} $k (k \in N)$ функции $f(z)$, если $k$
– максимальная по модулю степень у ненулевого члена главной части лорановского разложения в проколотой окрестности точки $z_0$. А именно, ряд (1) имеет коэфф $c_{-k}\neq 0$,
в то время как $c_{-n}$ = 0 $\forall n > k$.

    4)Точка $z_0$ – \textbf{существенно особая точка} $f(z)$, если
главная часть ряда (1) содержит бесконечное число членов.

\textbf{Теорема 2.} Следующие 3 утверждения эквивалентны: a) $z_0$ - устранимая особая точка функции $f(z)$, б) $\exists$ конечный $\displaystyle\lim_{z\rightarrow z_0}f(z)$, в) $f(z)$ ограничена в некоторой окрестности точки $z$.

$\blacktriangleright\;$
    a)$\rightarrow$б): По условию $f(z)=\sum_{n=0}^{\infty}c_n(z-z_0)^n$, $z \in D$. Сумма $g(z)$ стоящего справа ряда непрерывна в т.$z_0$ и ее значение в этой точке равно свободному члену $c_0$ ряда, т.к вне $z_0$ функции $f(z)$ и $g(z)$ совпадают, то $\exists \displaystyle\lim_{z\rightarrow z_0}f(z)=c_0$.
    \noindentб)$\rightarrow$в) функция имеющая конечный $lim$ в точке $z_0$ ограничена в некоторой окрестности этой точки.
    \noindentв)$\rightarrow$а) По условию в некоторой окрестности $U$ точки $z_0$ выполняется соотношение $|f(z)|\leq M \forall z \in U$. $\sqsupset$ $\gamma:|z-z_0|=\rho$ - окружность принадлежащая этой окрестности. Как $\Rightarrow$ из доказательства т.Лорана коэффициенты ряда (1) представимы в виде: $c_n = \frac{1}{2\pi i} \oint_{\gamma} \frac{f(z)}{(z-z_0)^{n+1}} \,dz \Rightarrow |c_n|\leq M\rho^{-1}$. Для отрицательных $n$ правая часть этой оценки стремится к 0 при $\rho \rightarrow 0$. Таким образом в ряде (1) все коэффициенты $c_n$ с отрицательными индексами $=0\Rightarrow z_0$ устранимая особая точка $f(z) \blacksquare$. 
    
\textbf{Теорема 3.} Изолированная особая точка $z_0$ функции $f(z)$ является ее полюсом $\Longleftrightarrow \displaystyle\lim_{z\rightarrow z_0}f(z)=\infty$.

$\blacktriangleright\;$
    1)$\sqsupset$ $z_0$ – полюс $f(z)$, тогда в некоторой проколотой окрестности K точки z0 имеется представление $f(z)=\frac{c_{-k}}{(z-z_0)^k}+...+c_0+c_1(z-z_0)+...(3)$,
    где $c_{-k}\neq0$. Равенство (3) можно переписать в виде:
    $f(z)(z-z_0)^k=c_{-k}+c_{-k+1}(z-z_0)+...+c_0(z-z_0)^k+...$, причем ряд, стоящий в правой части последнего равенства, сходится в некотором круге $K_r=\{z:|z-z_0|<r\}$. Если $\phi(z)$ сумма этого ряда, то $\phi(z) \in A(K_r)$, $\phi(z_0)=c_{-k}\neq0$. Поэтому $f(z)=\frac{\phi(z)}{(z-z_0)^k}$ и очевидно $\displaystyle\lim_{z\rightarrow z_0}f(z)=\infty$.
    2)Обратно, $\sqsupset$ $\displaystyle\lim_{z\rightarrow z_0}f(z)=\infty$. Тогда существует
    проколотая окрестность $K$ точки $z_0$, где $f(z) \neq 0$, поэтому
    в $K$ определена аналитическая функция $g(z) = \frac{1}{f(z)}$, причём справедливо представление: $g(z)=a_k(z-z_0)^k+a_{k+1}(z-z_0)^{k+1}+...=(z-z_0)^k(a_k+a_{k+1}(z-z_0)+...)$, где $k\geq1$, $a_k\neq0$. Значит $g(z)=(z-z_0)^k\phi(z)$, где $\phi(z_0)\neq0$. Тогда $f(z)=\frac{1}{g(z)}=\frac{1}{(z-z_0)^k}\frac{1}{\phi(z)}=\{\phi(z) \in A(K) \Rightarrow \frac{1}{\phi(z)} \in A(K) \text{ значит можно разложить в ряд Лорана}\}=
    \frac{1}{(z-z_0)^k}(b_0+b_1(z-z_0)+...)$, где $b_0=\frac{1}{\phi(z_0)}=\frac{1}{a_k}\neq0$, т.e $z_0$ -- полюс $f(z) \blacksquare$.
    
\textbf{Теорема 4.} Точка $z_0$ – полюс порядка $k$ функции $f(z)$ $\Longleftrightarrow$ в K справедливо представление: $f(z)=\frac{\phi(z)}{(z-z_0)^k}$, где $\phi(z)\in A(z_0)$, $\phi(z_0)\neq 0$.

\textbf{Теорема 5.} Изолированная особая точка $z_0$ функции
 $f(z)$ является существенно особой $\Longleftrightarrow$ не $\exists$ $\displaystyle\lim_{z\rightarrow z_0}f(z)$.

\textbf{Теорема 6.(Сохоцкого)} $\sqsupset$ $z_0$ – существенно особая точка функции $f(z)$. Тогда
для произвольного числа $A \in \mathbb{C}$ найдётся такая послед $\{z_n\}\rightarrow z_0:f(z_n)\rightarrow A$, $n \rightarrow \infty$.


% -------- source --------
[\cite{tfkp_stud_ticket}]
[\cite{tfkp_msu}]
