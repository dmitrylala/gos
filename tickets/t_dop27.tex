\setcounter{section}{4}
\setcounter{subsection}{27}
\setcounter{equation}{0}
\textbf{\LARGE dop 27. Разностная аппpоксимация задачи Диpихле для уpавнения Пуассона: постановка  pазностной задачи, оценка погpешности.}

Рассмотрим задачу Дирихле для уравнения Пуассона в прямоугольнике $G = \{ 0 < x_1 < l_1, \; 0< x_2 <l_2\}$:
\begin{equation}\label{27.1}
\begin{cases}
    \Delta u(x) = - f(x), \quad x = (x_1, x_2) \in G,\\
    u(x) = \mu(x), \quad\quad\;\; x \in \partial G.
\end{cases}
\end{equation}

Введем в $\overline{G}$ равномерную разностную сетку $\Omega_h = \omega_h \cup \gamma_h$:
$$
x_{ij} = (x_{1,i},\, x_{2,j});\;\; x_{1,i} = i h_1, \; x_{2,j} = j h_2; \;\; h_1 N_1 = l_1, \; h_2 N_2 = l_2;
$$
$$
\omega_h = \{x_{ij}; \;\; i = 1, 2, \ldots, N_1 -1; \;\; j = 1, 2, \ldots, N_2-1\};
$$
$$
\gamma_h =\{x_{i0}, \, x_{iN_2}, \, x_{0j}, \, x_{N_1 j};\;\; i = 1, 2, \ldots, N_1 -1; \;\; j = 1, 2, \ldots, N_2-1 \}.
$$
Здесь $\omega_h$ -- множество внутренних, а $\gamma_h$ -- множество граничных узлов сетки. Будем пользоваться обозначениями
$$
y_{\overline{x}_1x_1, \, ij} = \frac{y_{i-1 j} - 2 y_{ij} + y_{i+1 j}}{h_1^2}, \quad
y_{\overline{x}_2x_2, \, ij} = \frac{y_{i j-1} - 2 y_{ij} + y_{i j+1}}{h_2^2},
$$
$$
\Delta_h y_{ij} = (y_{\overline{x}_1x_1} + y_{\overline{x}_2x_2})_{ij} \; \textit{-- пятиточ. разностный оп-р Лапласа},
$$
где $y_{ij} = y(x_{ij})$ -- сеточная функция, определенная на $\Omega_h$. Сопоставим задаче \eqref{27.1} разностную схему:
\begin{equation}\label{27.2}
    \begin{cases}
        \Delta_h y_{ij} = - f(x_{ij}), \quad x_{ij} \in \omega_h,\\
        y_{ij} = \mu(x_{ij}). \quad\quad x_{ij} \in \gamma_h.
    \end{cases}
\end{equation}
Запишем уравнение $\Delta_h y_{ij} = - f(x_{ij})$ в виде
$$
\left(\frac{2}{h_1^2} + \frac{2}{h_2^2}\right) y_{ij} = \frac{y_{i-1 j} + y_{i+1 j}}{h_1^2} + \frac{y_{i j-1} + y_{i j+1}}{h_2^2} + f_{ij}, \quad x_{ij} \in \omega_h.
$$
Введем обозначения:
\begin{itemize}
    \item $x = x_{ij}$ -- центральный узел шаблона;
    \item $\Sh(x) = \{x,\, x_{i\pm1 j},\, x_{i j \pm 1} \}$ -- шаблон уравнений;
    \item $\Sh'(x) = (x) \setminus \{x\}$ -- окрестность узла $x$;
    \item $A(x) = \frac{2}{h_1^2} + \frac{2}{h_2^2}$, $B(x, x_{i\pm1 j}) = \frac{1}{h_1^2},  B(x, x_{i j\pm1}) = \frac{1}{h_2^2}, F(x) = f(x_{ij})$.
\end{itemize}
В этих обозначениях систему \eqref{27.2} можно привести к \textbf{канонической форме записи}
$$
\begin{cases}
    A(x) y(x) = \sum_{\xi \in \Sh'(x)} B(x, \xi) y(\xi) + F(x), \quad x\in \omega_h,\\
    A(x) y(x) = F(x), \quad x\in \gamma_h, \quad \textit{где} \; \Sh'(x) = \varnothing, \; A(x) = 1,\; F(x) = \mu(x).
\end{cases}
$$
\textbf{Замечание:\;} выполнены \textit{условия положительности коэффициентов}
$$
A(x) > 0, \;\; B(x, \xi) > 0, \;\; D(x) = A(x) - \sum_{\xi \in \Sh'(x)} B(x, \xi) \geq 0,\;\; x\in \Omega_h.
$$
\rule{275pt}{0.5pt}

\textbf{Принцип максимума для разностных схем}

\fbox{\textbf{Опр. 1}}\; Пусть функции $A(x), \; B(x, \xi), \; F(x)$ определены при всех $x, \; \xi \; \in \Omega_h$. \textbf{Разностной схемой в канонической форме записи} называется СЛАУ относительно неизвестной сеточной функции $y(x)$, определенной на $\Omega_h$,
$$
A(x) y(x) =  \sum_{\xi \in \Sh'(x)} B(x, \xi) y(\xi) + F(x), \quad x\in \Omega_h,
$$
если каждому узлу $x \in \Omega_h$ сопоставлен один и только один шаблон и одно и только одно уравнение.

\fbox{\textbf{Опр. 2}}\; Сетка $\Omega_h$ называется \textbf{связной}, если
$$
\forall \, x', x'' \in \Omega_h:\; \Sh'(x') \ne \varnothing\;\; \exists x_i \in \Omega_h, i = 1, 2, \ldots, m:
$$
$$
x_1 \in \Sh'(x'),\; x_2 \in \Sh'(x_1), \ldots, x_m \in \Sh'(x_{m-1}), \; x'' \in \Sh'(x_m).
$$
\fbox{\textbf{Опр. 3}}\; В узле $x\in\Omega_h$ выполнены \textbf{условия положительности коэффициентов}, если
\begin{equation}\label{27.3}
    A(x) > 0, \; B(x, \xi) > 0, \;\forall \xi \in \Sh'(x), \; D(x) = A(x) - \sum_{\xi \in \Sh'(x)} B(x, \xi) \geq 0.
\end{equation}

Определим линейный оператор $L$ формулами $$
L y(x) = A(x) y(x) - \sum_{\xi \in \Sh'(x)} B(x, \xi) y(\xi), \quad x\in \Omega_h.$$

Далее считаем, что $\Omega_h = \omega_h \cup \gamma_h$, где $\omega_h \ne \varnothing$ -- множество внутренних узлов, $\gamma_h$ -- множество граничных узлов (м.б. пустым).

\textbf{Теорема 1 \textit{(принцип максимума)}:\;} Пусть функция $y(x)$ определена и не является постоянной на связной сетке $\Omega_h$, а условия \eqref{27.3} выполнены при всех $x \in \omega_h$. Тогда если $L y(x) \leq 0$ ($L y (x) \geq 0$) для любых $x \in \omega_h$, то $y(x)$ не может принимать наибольшего положительного (наименьшего отрицательного) значения на $\omega_h$ среди всех своих значений на $\Omega_h$.

$\blacktriangleright\;$ От противного. Пусть $\,\exists  \, x'\in \omega_h: \; y(x') = \max_{x\in\Omega_h} y(x) > 0$, тогда $\,\exists\, x''\in\Omega_h:\; y(x'') < y(x')$, поскольку $y(x) \ne const$ на $\Omega_h$. В силу связности $\Omega_h$:
\begin{align*}
    &\exists \, x_i \in \omega_h, \; i = 1, \ldots, m:\\
    & x_1 \in \Sh'(x'), \; x_2 \in \Sh'(x_1), \ldots, x_m \in \Sh'(x_{m-1}), \; x'' \in \Sh'(x_m).
\end{align*}
В силу условий \eqref{27.3} и неравенства $y(\xi) \leq y(x') = \max_{x\in\Omega_h} y(x), \; \forall \xi \in \Omega_h$
$$
L y(x') = D(x') y(x') +  \sum_{\xi \in \Sh'(x')} B(x', \xi) (y(x') - y(\xi)) \geq 0.
$$
В то же время по условию $Ly(x') \leq 0$, т.к. $x' \in \omega_h$. Отсюда вытекает, что
$$
L y(x') = 0, \; y(\xi) = y(x'), \;\forall \xi \in \Sh'(x') \implies y(x_1) = y(x') = \max_{x\in\Omega_h} y(x) > 0.
$$
Аналогично покажем, что $y(x_2) = y(x_1) = \max_{x\in\Omega_h} y(x) > 0$, и так далее. Окончательно получим $y(x'') = y(x_m) = \ldots =y(x_1) = y(x')$, что противоречит неравенству $y(x'') < y(x')$. $\;\blacksquare$

Далее будем считать сетку $\Omega_h$ связной и выполненным условие
\begin{equation}
    \label{27.4}
    \exists\, x_0 \in \Omega_h:\; D(x_0) > 0.
\end{equation}

\textbf{Следствие 1 \textit{($\exists\,!$ решения)}:}\; Пусть выполнены условия \eqref{27.3} при всех $x \in \Omega_h$ и \eqref{27.4}. Тогда разностная задача $L y(x) = F(x), \; x\in \Omega_h$ имеет единственное решение.

\textbf{Следствие 2 \textit{(теорема сравнения)}:}\; Пусть выполнены условия \eqref{27.3} при всех $x \in \Omega_h$ и \eqref{27.4}. Тогда если $|F(x)| \leq \overline{F}(x)$ для любых $x \in \Omega_h$, то $|y(x)| \leq \overline{y}(x)$ для любых $x \in \Omega_h$, где $y(x)$ -- решение задачи $L y(x) = F(x), \; x\in \Omega_h$, $\overline{y}(x)$ -- решение задачи $L\overline{y}(x) = \overline{F}(x),\; x\in\Omega_h$.

\rule{275pt}{0.5pt}

\textbf{Устойчивость и сходимость разностной задачи \eqref{27.2}}

По следствию 1 имеем существование и единственность решения задачи \eqref{27.2}. Представим это решение в виде $y(x) = y_\mu(x) + y_F(x)$, где
\\$y_\mu(x)$ -- решение задачи $L y_\mu(x) = 0, \;x \in \omega_h; \;\; y_\mu(x) = \mu(x),\;x\in\gamma_h$,
\\$y_F(x)$ -- решение задачи $L y_F(x) = F(x), \; x\in \omega_h;\;\;y_F(x) = 0, \; x\in\gamma_h$.

\textbf{Теорема 2 \textit{(устойчивость по граничным условиям)}:}
$$\|y_\mu (x)\|_{C(\Omega_h)} \leq \|\mu(x)\|_{C(\gamma_h)}.$$

$\blacktriangleright\;$ Из принципа максимума следует $ \max_{x\in\omega_h} y(x) \leq \max_{x\in\gamma_h} \mu(x)$. Так как $y_\mu(x) = \mu(x),\;x\in\gamma_h$, то $ \max_{x\in\Omega_h} y(x) \leq \max_{x\in\gamma_h} \mu(x)$.\; $\blacksquare$

\textbf{Теорема 3 \textit{(устойчивость по правой части)}:}
$$\|y_F(x)\|_{C(\Omega_h)} \leq \frac{l_1^2 + l_2^2}{4} \|F(x)\|_{C(\omega_h)}.$$

$\blacktriangleright\;$ Рассмотрим функцию $\overline{y}(x) = K(l_1^2 + l_2^2 - x_1^2 - x_2^2),\; K= const, \; x = (x_1, x_2) \in \Omega_h$. Учтем, что на многочленах второй степени вторая разностная производная имеет те же значения, что и дифференциальная производная.
Поэтому $L \overline{y}(x) = - \Delta_h \overline{y}(x) = -(y_{\overline{x}_1x_1} + y_{\overline{x}_2x_2})_{ij} = 4K$. Положим $K = \frac{1}{2}\|F(x)\|_{C(\omega_h)}$, тогда $\overline{y}(x)$ является решением задачи
$$
L\overline{y}(x) = \overline{F}(x), \; x\in \omega_h, \quad \overline{y}(x) = \overline{\mu}(x),\; x\in \gamma_h,
$$
где $\overline{F}(x) = \|F(x)\|_{C(\omega_h)}, \; \overline{\mu}(x) > 0$. По следствию 2 имеем $|y_F(x)| \leq \overline{y}(x), \;\forall x \in \Omega_h$, то есть
$$
\|y_F(x)\|_{C(\Omega_h)} = \max_{x\in\Omega_h} |y_F(x)| \leq \max_{x\in\Omega_h} |\overline{y}(x)| \leq   \frac{l_1^2 + l_2^2}{4} \|F(x)\|_{C(\omega_h)}. \;\;\blacksquare
$$
\textbf{Теорема 4 \textit{(устойчивость разностной задачи Дирихле)}:} \; Для решения $y(x)$ разностной задачи \eqref{27.2} справедлива оценка
$$
\|y(x)\|_{C(\Omega_h)} \leq \|\mu(x)\|_{C(\gamma_h)} + \frac{l_1^2 + l_2^2}{4} \|F(x)\|_{C(\omega_h)}.
$$

\textbf{Теорема 5 \textit{(сходимость разностной задачи Дирихле)}:}\; Пусть $u$ -- решение дифференциальной задачи \eqref{27.1}, а $y$ -- решение разностной задачи \eqref{27.2}. Тогда $\|y - u\|_{C(\Omega_h)} \leq M(h_1^2 + h_2^2)$, где $M$ -- постоянная, не зависящая от $h_1, \; h_2$.

$\blacktriangleright\;$ Погрешность разностной задачи \eqref{27.2} $z_{ij} = y_{ij} - u_{ij}, \; x_{ij} \in \Omega_h$ удовлетворяет уравнениям
$$
\begin{cases}
    \Delta_h z_{ij} = - \psi_{ij}, \quad x_{ij} \in \omega_h,\\
    z_{ij} = 0, \qquad\quad x_{ij} \in \gamma_h,
\end{cases}
$$
где $\psi_{ij} = (u_{\overline{x}_1x_1} + u_{\overline{x}_2x_2} + f)_{ij}$ -- погрешность аппроксимации разностной задачи \eqref{27.2} на решении $u(x)$ дифференциальной задачи \eqref{27.1}. Поскольку вторая разностная производная аппроксимирует вторую разностную производную достаточно гладкой функции со вторым порядком, справедливо неравенство
$
\|\psi\|_{C(\omega_h)} \leq \widetilde{M}(h_1^2 + h_2^2),
$
где постоянная $\widetilde{M}$ не зависит от $h_1, \;h_2$. В силу устойчивости имеем
$$
\|z(x)\|_{C(\Omega_h)} \leq \frac{l_1^2 + l_2^2}{4} \|\psi\|_{C(\omega_h)} \;\implies \;
\|z(x)\|_{C(\Omega_h)} \leq M(h_1^2 + h_2^2). \; \blacksquare
$$


% -------- source --------
\bigbreak
[\cite[pages 32-41]{chmmf}]
