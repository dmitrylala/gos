\setcounter{section}{4}
\setcounter{subsection}{9}
\setcounter{equation}{0}
\textbf{\LARGE dop 9. Теорема Рисса о представлении линейного функционала.}\\


\textbf{Лемма}.\\
Пусть линейный ограниченный функционал $f(x)$, $x \in H$, не является
аннулирующим (т.е. $ \exists x \in H : f(x) \neq 0$), тогда $dim (coker f) = 1$.\\

\textbf{Доказательство}:\\
Очевидно, $ ker f \neq H$ (иначе $f \equiv 0$). Пусть $x1, x2 \not\in ker f$, тогда\\
$f(x_{2})x_{1} - f(x_{1})x_{2} \in ker f$, стало быть, $0 < dim(coker f) < 2$
, что и требовалось доказать.\\

\textbf{Теорема Рисса о представлении линейного ограниченного функционала в
гильбертовом пространстве}:\\
Для любого линейного ограниченного функционала $f(x)$, $x \in H$,
существует и притом единственный элемент $h \in H$ такой, что $f(x)= (x, h)$, причем $||f|| = ||h||$.\\

\textbf{Доказательство}:\\
Если $f$ аннулирующий, то $h = 0$; иначе в силу леммы
$dim(ker f)^{\perp} = 1$. Очевидно, $ker f$ – замкнутое линейное подпространство. По теореме о
разложении гильбертова пространства $H = ker f + (ker f)^{\perp}$, так что $\forall x \in H, \exists! \space x_{1} \in ker f$,
$x_{2} \in (ker f)^{\perp} : x = x_{1} + x_{2}$. Но тогда $f(x) = f(x_{2})$. Пусть $(ker f)^{\perp} = L(e), ||e|| = 1$, тогда в силу следствия из теоремы о разложении $x_{2} = (x, e)e$, стало быть, $f(x) = (x,e)f(e) = (x, h)$, где $h = \overline{f(e)} e$.\\
Далее, по неравенству Коши-Буняковского $|f(x)| < ||x|| ||h||$, откуда следует, что $||f|| \leq ||h||$. Положив $x = h / ||h||$, получим противоположное неравенство, откуда следует, что $||f|| = ||h||$.\\
Единственность: если $f(x) = (x, h) = (x, \tilde h)$, то $(x, h - \tilde h) = 0$, $\forall x \in H$, и осталось положить $x = h - \tilde h$ . Теорема доказана.\\

Следствие 1 Соответствие между $f$ и $h$ взаимно однозначное и изометричное.\\
Следствие 2 Оператор соответствия $f \rightarrow h$ антилинейный.\\
Следствие 3 $H \cong H^{*} \cong H^{**}$
% -------- source --------
\bigbreak
[\cite{funcan_spring}, стр. 2]
