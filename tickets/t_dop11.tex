\textbf{\LARGE dop 11. Компактные операторы.}

\textbf{Доп. теория}


Метрическое пространство $(X,\rho)$ называется \textbf{предкомпактным} ($X$ - множество, $\rho$ - метрика на нем), если у любой последовательности в $X$ существует фундаментальная подпоследовательность.
\bigbreak

Метрическое пространство $(X,\rho)$ называется \textbf{ограниченным}, если $\sup_{x,y \in X}{\rho (x,y)} < \infty$.

Предкомпактное $(X,\rho)$ ограничено (т.к $\rho(x_m,y_n) \rightarrow \infty$ противоречит фундаментальности $\{x_m\}$ и $\{y_n\}$).
\bigbreak

Полное евклидово (унитарное) пространство называется \textbf{гильбертовым}. \\
Полное линейное нормированное пространство называется \textbf{банаховым}.
\bigbreak 

\textbf{Слабая сходимость} $\{x_n\} \xrightarrow[n \rightarrow \infty]{w} x$ в $H$ означает, что $\lim_{n \rightarrow \infty}{(x_n, h)} = (x, h), \ \forall h \in H$ - гильбертово пространство.
\bigbreak

Оператор $A:X \rightarrow Y$ называется \textbf{непрерывным} или \textbf{ограниченным}, если он переводит ограниченное множество в ограниченное.
\bigbreak



$LB(X,Y)$ - множество всех линейных ограниченных операторов, действующих из $X$ в $Y$.
\bigbreak

$\boldsymbol{\varepsilon}$\textbf{-сеть} для подмножества $M \in X$ метрического пространства $(X, \rho)$, есть множество $Z \in X:$
$$
    \forall x \in M \ \exists \ z \in Z: \rho(x,z) < \varepsilon.
$$

$M$ - \textbf{вполне ограниченное}, если для $\forall \varepsilon > 0$ его можно накрыть конечной $\varepsilon$-сетью. 
$$
    M \subset \bigcup_{k=1}^{n} B(x_k, \varepsilon), \ 
$$
где $B(x_k, \varepsilon)$ - шар радиуса $\varepsilon$ с центром в $x_k$.
\bigbreak

\textbf{Теорема Хаусдорфа}. Метрическое пространство $M$ - предкомпактно $\Leftrightarrow$ оно вполне ограничено.
\bigbreak

\textbf{Билет} \\

\mathLet $X$ и $Y$ - банаховы пространства. Линейный оператор $A: X \rightarrow Y$ называется \textbf{компактным}, если он любое ограниченное множество из $X$ переводит в множество, предкомактное в $Y$.
\bigbreak

Оператор называется \textbf{вполне непрерывным}, если он любую слабо сходящуюся последовательность переводит в последовательность, сходящуюся по норме.
\bigbreak

\textbf{Теорема.} Линейный оператор вполне непрерывный $\Leftrightarrow$ он
компактный.
\begin{proof}
    $\Rightarrow$ $A$ - вполне непрерывный. Возьмем ограниченную последовательность $\{x_n\}$. Из любой ограниченной последовательности можно выбрать
слабо сходящуюся подпоследовательность $\{x_{n_k}\}$. $Ax_{n_k} \rightarrow Ax$, так как $A$ - вполне непрерывный. Следовательно $A$ переводит огрниченную п-ть $\{x_n\}$ в п-ть $\{Ax_n\}$, у которой можно выбрать фундаментальную подп-ть  $\{Ax_{n_k}\}$. Следовательно $A$ - компактный.

    $\Leftarrow$ От противного. $A$ - компактный. Пусть $x_n \xrightarrow{w} x, Ax_n \nrightarrow  Ax$. Тогда $\exists \varepsilon > 0 \text{ и } \exists \{n_k\}:$
    $$
        ||Ax_{n_k} - Ax|| \ge \epsilon.
    $$

    Так как $\{x_{n_k}\}$ сходится слабо, то она ограничена. Так как $A$ - компактный, то $\{Ax_{n_k}\}$ - предкомпактна, т.е из нее можно выбрать фундаментальную подпоследовательность, что противоречит неравенству.
\end{proof}

\bigbreak
\textbf{Примеры компактных (вполне непрерывных) операторов.}
\begin{enumerate}
    \item Если пространства $X$ и $Y$ конечномерные, то любой линейный оператор ограничен, т.е. переводит ограниченное множество в ограниченное, но любое ограниченное множество предкомпактно в конечномерном пространстве, Таким образом, в конечномерных пространствах все линейные операторы компактны.

    \item Для нулевого оператора образом является одна точка, значит он компактен.

    \item Пусть $X$ и $Y$ - произвольные нормированные пространства. Оператором \textbf{конечного ранга} называется $A \in LB(X,Y)$, если его образ $Im A$ является конечномерным пространством. \\
    Покажем, что операторы конечного ранга являются компактными. Если множество $M \subset X$ ограниченное, то $A(M) \subset Y$ ограничено и в силу конечномерности $Im A$ множество $A(M)$ предкомпактно.

    \item Интегральный оператор с вырожденным ядром. \\
    В пространстве $C[\,0,1]\,$ рассмотрим интегральный оператор с вырожденным ядром, т.е.
    $$
        Ax(t) = \int_0^1 K(t,s)x(s) \, ds,
    $$
    где $K(t,s) = \sum_{k=1}^n a_k(t)b_k(s)$, где $a_k(t), b_k(s)$ - непрерывные функции. \\
    Тогда
    $$
         Ax(t) = \sum_{k=1}^n a_k(t) \int_0^1 b_k(s)x(s) \, ds = \sum_{k=1}^n c_k a_k(t),
    $$
    т.е. образ $Im A$ принадлежит конечномерному пространству $L$, порожденному функциями $a_k(t)$. Интегральный оператор $A$ - ограниченный, следовательно, это оператор конечного ранга и он компактен.
\end{enumerate}

\bigbreak
\textbf{Свойства компактных операторов}
\begin{enumerate}
    \item Если $A$ и $B$ - \textbf{компактные операторы}, то оператор $\alpha A+ \beta B$ компактен ($\alpha, \beta \in \mathbb{R}$).
    \begin{proof}
        \begin{enumerate}
            
            \item Докажем, что $A+B$ - компактен. \\ 
            \mathLet $M$ - ограниченное множество. В образе $(A+B)(M)$ возьмем последовательность $y_n = (A+B)x_n$. В силу компактности оператора $A$ из $Ax_n$ можно выделить сходящуюся подпоследовательность $Ax_{n_k}$, а из подпоследовательности $Bx_{n_k}$, в силу компактности B - сходящуюся подпоследовательность $Bx_{n_{k_i}}$. Подпоследовательность $(A+B)x_{n_{k_i}}$ сходится, значит, множество $(A+B)(M)$ предкомпактно (т.к. для любой п-ти $y_n$ существует сходящаяся подп-ть) и оператор $A+B$ компактен.

            \item Оператор умножения на число $\lambda$ ограничен, и, значит $\lambda A$ - компактный.

            \item Учитывая предыдущие пункты, любая линейная комбинация компактных операторов дает компактный оператор.
        \end{enumerate}
        
    \end{proof}

    \item Произведение компактного и ограниченного операторов (в любом порядке) - компактный оператор.


    \item Если $A \in LB(X,Y)$, $\{An\}$ – последовательность компактных операторов, действующих из $X$ в $Y$ и $||A_n - A|| \rightarrow 0$, то $A$ - компактный.
    \begin{proof}
        Пусть $M$ - ограниченное множество в $X$ и $||x|| \le C \text{ для } x \in M$.
        Воспользуемся теоремой Хаусдорфа. Для $\forall \varepsilon > 0$ рассмотрим конечную $\varepsilon$-сеть для множества $A(M)$. Выберем $n_0: ||A_{n_0} - A|| \le \varepsilon/2C$. Т.к. множество $A_{n_0}(M)$ предкомпактно, то для него существует конечная $\varepsilon/2$-сеть $S = (s_1, ..., s_m).$ Покажем, что $S - \varepsilon$-сеть для $A(M)$. Пусть $y \in A(M)$. т.е. $y = Ax, \ x \in M$. Существует $s_i: ||s_i - A_{n_0}x|| \le \varepsilon/2$ (определение $\varepsilon$-сети для $A_{n_0}(M)$). Тогда
        \begin{align*}
            ||y-s_i|| \le ||y - A_{n_0}x|| + ||A_{n_0}x - s_i|| \le \\
            \le ||A - A_{n_0}|| ||x|| + \varepsilon/2 \le (\varepsilon/2C)C + \varepsilon/2 = \varepsilon .
        \end{align*}
    \end{proof}
\end{enumerate}

[\cite{funcan_spring}]
[\cite{funcan_antonevich}]
