\textbf{\LARGE dop 22. Пpимеpы и канонический вид одношаговых итеpационных методов pешения систем линейных алгебpаических уpавнений..}


\par Рассмотрим систему $Ax=f$, где матрица $A=\begin{bmatrix}a_{ij}\end{bmatrix}$, $i$, $j=1$, $2$,..., $m$ имеет обратную, $x=\left( x_{1},\ldots ,x_{m}\right) ^{T}$, $f=\left( f_{1},\ldots ,f_{m}\right) ^{T}$.
\par Рассм. примеры итерац. методов. Преобразуем $Ax=f$ к $x_{i}=-\sum ^{i-1}_{j=1}\dfrac{a_{ij}}{a_{ii}}x_{j}-\sum ^{m}_{j=i+1}\dfrac{a_{ij}}{a_{ii}}x_{j}+\dfrac{f_{i}}{a_{ii}}$, $i=1$, $2$,..., $m$, $a_{ii}\neq 0$. Пусть значение суммы равно нулю, если верхний предел суммирования меньше нижнего. Тогда при $i=1$: $x_{1}=-\sum ^{m}_{j=2}\dfrac{a_{1j}}{a_{11}}x_{j}+\dfrac{f_{1}}{a_{11}}$. В дальнейшем верхний индекс это номер итерации, например: $x^{n}=\left( x_{1}^{n},\ldots ,x_{m}^{n}\right) ^{T}$, где $x_{i}^{n}$- $n$-ая итерация $i$-ой компоненты $\overrightarrow{x}$.
\par В методе Якоби итерации определяются:
\begin{equation*}
   x_{i}^{n+1}=-\sum ^{i-1}_{j=1}\dfrac{a_{ij}}{a_{ii}}x_{j}^{n}-\sum ^{m}_{j=i+1}\dfrac{a_{ij}}{a_{ii}}x_{j}^{n}+\dfrac{f_{i}}{a_{ii}} 
\end{equation*}
$n=0$, $1$,..., $n_{0}$, $i=1$, $2$,..., $m$. Начальные значения $x_{i}^{0}$, $i=1$, $2$,..., $m$ задаются произвольно. Окончание итераций определяется либо заданием максимального числа итераций $n_{0}$, либо условием: $\max \left| x_{i}^{n+1}-x_{i}^{n}\right| <\epsilon$ при $1\leq i\leq m$, $\epsilon>0$- заданное число.
\par Итерационный метод Зейделя: 
\begin{equation*}
    x_{i}^{n+1}=-\sum ^{i-1}_{j=1}\dfrac{a_{ij}}{a_{ii}}x_{j}^{n+1}-\sum ^{m}_{j=i+1}\dfrac{a_{ij}}{a_{ii}}x_{j}^{n}+\dfrac{f_{i}}{a_{ii}}
\end{equation*}
$n=0$, $1$,..., $n_{0}$, $i=1$, $2$,..., $m$. Распишем подробнее: $x_{1}^{n+1}=-\sum ^{m}_{j=2}\dfrac{a_{1j}}{a_{11}}x_{j}^{n}+\dfrac{f_{1}}{a_{11}}$, $x_{2}^{n+1}=-\dfrac{a_{21}}{a_{22}}x_{1}^{n+1}-\sum ^{m}_{j=3}\dfrac{a_{2j}}{a_{22}}x_{j}^{n}+\dfrac{f_{2}}{a_{22}}$. Первая компонента $x_{1}^{n+1}$ вектора $x^{n+1}$ находится из уравнения явным образом, для ее вычисления нужно знать $x^{n}$ и $f_1$. При нахождении $x_{2}^{n+1}$ используются найденное значение $x_{1}^{n+1}$ и известные значения $x_{j}^{n}$, $j=3$,..., $m$, с предыдущей итерации. Таким образом, компоненты $x_{i}^{n+1}$ вектора $x^{n+1}$ находятся последовательно, начиная с $i=1$.
\par Для исследования сходимости удобнее записывать методы в матричной форме. Представим матрицу $A$ системы $Ax=f$: $A=A_{1}+D+A_{2}$, где $D=diag\left[ a_{11},a_{22},\ldots,a_{mm}\right]$- диагональная матрица с той же главной диагональю, что и матрица $A$, $A_{1}$— нижняя треугольная и $A_{2}$— верхняя треугольная с нулевыми главными диагоналями. Представление $Ax=f$ в $x_{i}=-\sum ^{i-1}_{j=1}\dfrac{a_{ij}}{a_{ii}}x_{j}-\sum ^{m}_{j=i+1}\dfrac{a_{ij}}{a_{ii}}x_{j}+\dfrac{f_{i}}{a_{ii}}$ эквивалентно ее записи в виде матричного уравнения: $x=-D^{-1}A_{1}x-D^{-1}A_{2}x+D^{-1}f$.
\par Метод Якоби в векторной записи: 
\begin{equation*}
   x^{n+1}=-D^{-1}A_{1}x^{n}-D^{-1}A_{2}x^{n}+D^{-1}f 
\end{equation*}
%x^{n+1}=-D^{-1}A_{1}x^{n}-D^{-1}A_{2}x^{n}+D^{-1}f$ 
или 
\begin{equation*}
    Dx^{n+1}+\left( A_{1}+A_{2}\right)x^{n}=f
\end{equation*}
\par Метод Зейделя: 
\begin{equation*}
    x^{n+1}=-D^{-1}A_{1}x^{n+1}-D^{-1}A_{2}x^{n}+D^{-1}f
\end{equation*} 
или 
\begin{equation*}
    \left( D+A_{1}\right)x^{n+1}+A_{2}x^{n}=f
\end{equation*}
\par Тогда: $D\left( x^{n+1}-x^{n}\right)+Ax^n=f$, $\left( D+A_{1}\right)\left( x^{n+1}-x^{n}\right)+Ax^n=f$. 
\par Видно, что если итерационный метод сходится, то он сходится к решению исходной системы уравнений. Для ускорения сходимости вводят числовые параметры, которые зависят от номера итерации. Например, $D\frac{\left( x^{n+1}-x^{n}\right)}{\tau _{n+1}}+Ax^{n}=f$, $\left( D+A_{1}\right)\frac{\left( x^{n+1}-x^{n}\right)}{\tau _{n+1}}+Ax^{n}=f$.
\par Методы Якоби и Зейделя относятся к одношаговым итерационным методам, когда для нахождения $x^{n+1}$ требуется помнить только одну предыдущую итерацию $x^{n}$. Используются и многошаговые итерационные методы, в которых $x^{n+1}$ определяется через значения на двух и более предыдущих итерациях.
\par Теперь пусть $x_{n}$ будет обозначать вектор, полученный в результате $n$-ой итерации. 
\par Канонической формой одношагового итерационного метода решения $Ax=f$ называется: 
\begin{equation*}
    B_{n+1}\frac{\left( x_{n+1}-x_{n}\right)}{\tau _{n+1}}+Ax_{n}=f
\end{equation*}
$n=0$, $1$,..., $n_{0}$.  Здесь $B_{n+1}$ — матрица, задающая итерационный метод, $\tau _{n+1}$— итерационный параметр. Предполагается, что задано начальное приближение $x_{0}$ и $\exists$ матрицы $B_{n}^{-1}$, $n=1$,..., $n_{0}-1$ . Тогда можно последовательно определить все $x_{n}$, $n=1$,..., $n_{0}$. Для нахождения $x_{n+1}$ по известным $f$ и $x_{n}$: $B_{n+1}x_{n+1}=F_{n}$, где $F_{n}=\left( B_{n+1}-\tau_{n+1}A\right)x_{n}+\tau_{n+1}f$. 
\par Итерационный метод называют явным (неявным), если $B_{n}=E$
($B_{n}\neq E$), где $E$— единичная матрица. Неявные применяют, когда каждую $B_{n}$ обратить легче, чем исходную матрицу $A$. Например, в методе Зейделя приходится обращать треугольную матрицу.
\par Итерационный метод $B_{n+1}\frac{\left( x_{n+1}-x_{n}\right)}{\tau _{n+1}}+Ax_{n}=f$ называется стационарным, если $B_{n+1}=B$ и $\tau_{n+1}=\tau$ не зависят от номера итерации, и нестационарным в противопол. случае.
\bigbreak
\par Приведем еще примеры итерационных методов. 
\par Методом простой итерации называют явный метод $\frac{\left( x_{n+1}-x_{n}\right)}{\tau}+Ax_{n}=f$ с постоянным параметром $\tau$. 
\par Явный метод $\frac{\left( x_{n+1}-x_{n}\right)}{\tau_{n+1}}+Ax_{n}=f$ с переменным параметром $\tau_{n+1}$ называется итерационным методом Ричардсона. 
\par Обобщением метода Зейделя является метод верхней релаксации: $\left( D+wA_{1}\right)\frac{\left( x_{n+1}-x_{n}\right)}{w}+Ax_n=f$, где $w>0$ это заданный числовой параметр.
