\setcounter{section}{4}
\setcounter{subsection}{6}
\setcounter{equation}{0}
\textbf{\LARGE dop 6. Билинейные и квадpатичные фоpмы. Пpиведение их к каноническому виду. Закон инеpции.}


\begin{definition}
Пусть $V$~--- линейное пространство над $\bbR$. Отображение\\ $\mcA\colon V\times V\to \bbR$ называется \emph{билинейной формой} в пространстве $V$, если
\begin{enumerate}
    \item $\mcA(x+y,z)=\mcA(x,z)+\mcA(y,z)$,
    \item $\mcA(\alpha x,y)=\alpha\mcA(x,y)$,
    \item $\mcA(x,y+z)=\mcA(x,y)+\mcA(x,z)$,
    \item $\mcA(x,\alpha y)=\alpha\mcA(x,y)$,
\end{enumerate}
$\forall x,y,z\in V,~\alpha\in\bbR$.
\end{definition}

\begin{definition}
Билинейная форма называется \emph{симметрической}, если $\mcA(x,y)=\mcA(y,x),\forall x,y\in V$, и \emph{кососимметрической}, если $\mcA(y,x)=-\mcA(x,y),\forall x,y\in V$.
\end{definition}

\begin{example}
В $n$-мерном пространстве $V$ с базисом $e_1,\ldots,e_n$ отображение\\ $\mcA:V\times V\to\bbR$, определенное правилом
\begin{equation}
\label{eq общий вид билин. формы}
    \mcA(x,y)=\sum_{i,j=1}^na_{ij}x_iy_j,
\end{equation}
$\forall x=\sum_{i=1}^nx_ie_i,~y=\sum_{i=1}^ny_ie_i$, где $a_{ij}~(i,j=\overline{1,n})$~--- фиксированные числа, является билинейной формой (в силу линейности координат).
\end{example}

\begin{definition}
Представление билинейной формы в виде (\eqref{eq общий вид билин. формы}) называется \emph{общим видом билинейной формы в базисе $e$}.\\
Матрица $A_e=(a_{ij})\in\bbR^{n\times n}:a_{ij}=\mcA(e_i,e_j),~i,j=\overline{1,n}$ называется \emph{матрицей билинейной формы $\mcA(x,y)$ в базисе $e$}.\\
Общий вид билинейной формы может быть записан в компактном виде: если $x_e,~y_e$~--- координатные столбцы векторов $x$ и $y$ в базисе $e$, то
\begin{equation}
\label{eq комп. вид билин. формы}
    \mcA(x,y)=x_e^TA_ey_e,~\mcA(x,y)=y_e^TA_e^Tx_e.
\end{equation}
Первое из равенств проверяется непосредственно, второе можно получить транспонированием обеих частей первого.
\end{definition}

\begin{definition}
Пусть $\mcA(x,y)$~--- симметрическая билинейная форма в пространстве $V$ над полем $\bbP$. \emph{Квадратичной формой} называется отображение $\mcA\colon V\to\bbP$, которое $\forall x\in V\mapsto \mcA(x,x)$, то есть сужение симметрической билинейной формы на диагональ декартова квадрата $V\times V$.\\
\textsf{Обозначение:}~$\mcA(x,x)$ или $\mcA(x)$
\end{definition}

\begin{definition}
Билинейная форма $\mcA(x,y)$ при этом называется \emph{полярной билинейной формой} к квадратичной форме $\mcA(x,x)$.
\end{definition}

\begin{Commentwhite}
Из свойств билинейных форм следует:
\begin{enumerate}
    \item В базисе $e$ квадратичная форма $\mcA(x,x)$ с матрицей $A_e=(a_{ij})$ может быть записана в виде: $\forall x=\sum_{i=1}^nx_ie_i$
    \begin{equation}
    \label{eq общий вид кв. формы}
        \mcA(x,x)=\sum_{i,j=1}^na_{ij}x_ix_j,~a_{ij}=a_{ji},
    \end{equation}
    \item $rg\mcA(x,x)=rg\mcA(x,y)$.
\end{enumerate}
\end{Commentwhite}

\begin{definition}
Базис $e=(e_1,\ldots,e_n)$ называется \emph{каноническим базисом квадратичной формы} $\mcA(x,x)$, если её матрица в этом базисе диагональна, то есть $A_e=diag(\lam_1,\ldots,\lam_n)$.
\end{definition}

\begin{definition}
В каноническом базисе квадратичная форма $\mcA(x,x)$ имеет вид $\mcA(x,x)=\lam_1x_1^2+\ldots+\lam_nx_n^2$, который называется \emph{каноническим видом} квадратичной формы или \emph{суммой квадратов}. Числа $\lam_1,\ldots,\lam_n$ называются её \emph{каноническими} коэффициентами.
\end{definition}

\begin{Commentwhite}
Очевидно, число ненулевых квадратов совпадает с рангом $\mcA(x,x)$. Итак, если $e$~--- канонический базис и $r=rg\mcA(x,x)$, то
\begin{equation}
\label{eq канон. вид квадр. формы}
    \mcA(x,x)=\lam_1x_1^2+\ldots+\lam_rx_r^2,~\forall x=\sum_{i=1}^nx_ie_i.
\end{equation}
\end{Commentwhite}

\begin{theorem}
\label{th метод Лагранжа}
(Метод Лагранжа приведения к каноническому виду)\\
Для любой квадратичной формы существует канонический базис.
\end{theorem}

\begin{definition}
Пусть квадратичная форма $\mcA(x,x)$ приведена к каноническому виду (\eqref{eq канон. вид квадр. формы}). Число $\pi$ положительных квадратов в (\eqref{eq канон. вид квадр. формы}) и число $\nu=r-\pi$ называются \emph{положительным и отрицательным индексами инерции} квадратичной формы $\mcA$, а их разность $\sigma=\pi-\nu$ называется \emph{сигнатурой} $\mcA(x,x)$.
\end{definition}

\begin{theorem}
\label{th закон инерции}
(Закон инерции)\\
Положительный и отрицательный индексы инерции вещественной квадратичной формы не зависят от выбора канонического базиса.
\end{theorem}
\begin{proofocre}
~\\
Пусть $e$ и $f$~--- канонические базисы для $\mcA(x,x)$ ранга $r$ и для\\ $x=\sum_{i=1}^nx_ie_i=\sum_{i=1}^ny_if_i$
\begin{equation}
\label{eq 43.1}
\begin{gathered}
    \mcA(x,x)=a_1x_1^2+\ldots+a_px_p^2-a_{p+1}x_{p+1}^2-\ldots-a_rx_r^2,\\
    \mcA(x,x)=b_1y_1^2+\ldots+b_qy_q^2-b_{q+1}y_{q+1}^2-\ldots-b_ry_r^2,
\end{gathered}
\end{equation}
где $a_i>0,~b_i>0,~i=\overline{1,r}$.\\
Докажем, что $p\le q$. От противного: пусть $p>q$.\\
Рассмотрим подпространства $L_1=\mcL(e_1,\ldots,e_p)$ и $L_2=\mcL(f_{q+1},\ldots,f_n)$. Согласно $\dim (L_1+L_2) = \dim L_1 + \dim L_2 - \dim(L_1 \cap L_2)$, $dim(L_1\cap L_2)=p+(n-q)-dim(L_1+L_2);~dim(L_1+L_2)\le n,\\p>q \Rightarrow dim(L_1+L_2)>0 \Rightarrow\exists~x_0\neq\theta,~x_0\in L_1\cap L_2$.\\
Пусть $x_0=\alpha_1e_1+\ldots+\alpha_pe_p=\beta_{q+1}f_{q+1}+\ldots+\beta_nf_n$. Тогда, согласно (\eqref{eq 43.1})
\begin{equation}
    \mcA(x_0,x_0)=a_1\alpha_1^2+\ldots+a_p\alpha_p^2=-b_{q+1}\beta_{q+1}^2-\ldots-b_r\beta_r^2.
\end{equation}
Так как $x_0\neq\theta$, то $a_1\alpha_1^2+\ldots+a_p\alpha_p^2>0,~-b_{q+1}\beta_{q+1}^2-\ldots-b_r\beta_r^2<0 \Rightarrow$ (!) $\Rightarrow p\le q$.\\
Аналогично доказывается, что $p\ge q$. Значит, $p=q$.
\end{proofocre}


% -------- source --------
\bigbreak
[\cite{kim}]
