\setcounter{section}{4}
\setcounter{subsection}{24}
\setcounter{equation}{0}
\textbf{\LARGE dop 24. Интеpполяционная фоpмула Лагpанжа и оценка ее погpешности.}  \\

Пусть на отрезке $a \leq x \leq b$ заданы точки $x_k,$ $k=0,1,...,n$ (узлы интерполирования), в которых известны значения функции $f(x)$. Задача \textbf{интерполирования алгебраическими многочленами} состоит в том, чтобы построить многочлен
\begin{equation*}
    L_n(x)=a_0+a_1x+...+a_nx^n
\end{equation*}
степени $n$ которого в заданных точках $x_k$, $k=0,1,...,n,$ совпадают со значениями функции $f(x)$ в этих точках. \\
Для любой непрерывной функции $f(x)$ сформулированная задача имеет единственное решение. Действительно, для отыскания коэффициентов $a_0, a_1, ...,a_n$ получаем систему линейных уравнений
\begin{equation}
    \label{система уравнений}
    a_0 + a_1x_i+a_2x_i^2+...+a_nx_i^n=f(x_i), \quad i =0,1,...,n,
\end{equation}
определитель которой отличен от нуля, если среди точек $x_i$, $i=0,1,...,n$ нет совпадающих. \\
Многочлен $L_n(x_i)=f(x_i),$ удовлетворяющий условиям
\begin{equation}
    \label{условия интерполирования}
    L_n(x_i)=f(x_i), \quad i =0,1,...,n,
\end{equation}
называется \textbf{интерполяционным многочленом} для функции $f(x)$, построенным по узлам $\{x_i\}_0^n$. \\
Решение системы (\ref{система уравнений}) можно записать различным образом. \textbf{Интерполяционная формула Лагранжа} позволяет представить многочлен $L_n(x)$  в виде линейной комбинации
\begin{equation}
    \label{интерполяционная формула лагранжа}
    L_n(x) = \sum_{k=0}^{n} c_k(x) f(x_k)
\end{equation}
значений функции $f(x)$ в узлах интерполирования. \\
Найдем явное выражение для коэффициентов $c_k(x)$. Из условий интерполирования (\ref{условия интерполирования}) получаем
\begin{equation*}
    \sum_{k=0}^{n} c_k(x_i) f(x_k) = f(x_i), \quad i=0,1,...,n.
\end{equation*}
Эти соотношения будут выполнены, если на функции $c_k(x)$ наложить условия
\begin{equation*}
    c_k(x_i) =
    \begin{cases}
    0       & \quad i \neq k\\
    1  & \quad i = k, i =0,1,...,n,
  \end{cases}
\end{equation*}
которые означают, что каждая из функций $c_k(x)$, $k=0,1,...,n,$ имеет не менее $n$ нулей на $[a,b]$. Поскольку $L_n(x)$ - многочлен степени n, коэффициенты $c_k(x)$ естественно искать также в виде многочленов степени $n$, а именно в виде
\begin{equation*}
    c_k(x) = \lambda_k(x-x_0)(x-x_1)...(x-x_{k-1})(x-x_{k+1})...(x-x_n).
\end{equation*}
Из условия $c_k(x_k)=1$ находим
\begin{equation*}
    \lambda_k^{-1}=(x_k-x_0)(x_k-x_1)...(x_k-x_{k-1})(x_k-x_{k+1})...(x_k-x_n).
\end{equation*}
Таким образом, коэффициенты $c_k(x)$ интерполяционного многочлена (\ref{интерполяционная формула лагранжа}) находятся по формулам
\begin{equation*}
    c_k(x) = \frac{\displaystyle \prod_{j \neq k}(x-x_j)}{\displaystyle \prod_{j \neq k}(x_k-x_j)}
\end{equation*}

Часто коэффициенты $c_k(x)$ записывают в другом виде. Введем многочлен $\omega(x)$ степени $n+1$:
\begin{equation*}
    \omega(x) = (x-x_0)(x-x_1)...(x-x_{k-1})(x-x_k)(x-x_{k+1})...(x-x_n)
\end{equation*}
и вычислим его производную в точке $x_k$:
\begin{equation*}
    \omega '(x) = (x-x_k)...(x_k - x_{k-1})(x_k-x_{k+1})...(x_k-x_n).
\end{equation*}
Тогда получим, что
\begin{equation*}
    c_k = \frac{\omega (x)}{(x-x_k)\omega ' (x_k)}.
\end{equation*}
Итак, \textbf{интерполяционный многочлен Лагранжа} имеет вид
\begin{equation*}
    L_n(x) = \displaystyle \sum_{k=0}^{n} \frac{\omega(x)}{(x-x_k)\omega ' (x_k)} f(x_k)
\end{equation*}
или, более подробно,
\begin{equation*}
    L_n(x) = \displaystyle \sum_{k=0}^{n} \frac{\displaystyle \prod_{j \neq k}(x-x_j)}{\displaystyle \prod_{j \neq k}(x_k-x_j)} f(x_k)
\end{equation*}

\textbf{Остаточный член интерполяционной формулы}. Заменяя функцию $f(x)$ интерполяционным многочленом $L_n(x)$, мы допускаем погрешность
\begin{equation*}
    r_n(x) = f(x) - L_n(x),
\end{equation*}
которая называется \textbf{погрешностью интерполирования} или, что то же самое, \textbf{остаточным членом интерполяционной формулы}. Ясно, что в узлах интерполирования эта погрешность равна 0. Оценим погрешность в любой точке $x \in [a,b]$. Для этого рассмотрим вспомогательную функцию
\begin{equation*}
    g(s) = f(s) - L_n(s) - K\omega(s),
\end{equation*}
где $s \in [a,b],$ $K$ - постоянная и
\begin{equation}
    \label{вспомогательный многочлен}
    \omega(s) = (s-x_0)(s-x_1)...(s-x_n).
\end{equation}
Пусть требуется оценить $r_n(x)$ в заданной точке $x \in [a,b]$, не являющейся узлом интерполирования. Выберем постоянную $K$ из условия $g(x)=0.$ Для этого достаточно положить
\begin{equation*}
    K = \frac{f(x)-L_n(x)}{\omega(x)}.
\end{equation*}
Предположим, что $f(s)$ имеет $n+1$ непрерывную производную на отрезке $a \leq s \leq b.$ Функция $g(s)$ имеет не менее $n+2$ нулей на этом отрезке, а именно в точках $x, x_k,$ $k=0,1,...,n$. Поэтому производная $g'(s)$ имеет не менее чем $n+1$ нулей на $[a,b]$, $g''(s)$ - не менее $n$ нулей и т.д., функция $g^{(n+1)}(s)$ по крайней мере один раз обращается в нуль на $[a,b].$ Тем самым существует точка $\xi \in [a,b], $ в которой $g^{(n+1)}(\xi)=0$. \\
Поскольку
\begin{equation*}
    g^{(n+1)}(s) = f^{(n+1)}(s)-(n+1)!K,
\end{equation*}
получаем
\begin{equation*}
    f^{(n+1)}(\xi) = \frac{f(x)-L_n(x)}{\omega (x)} (n+1)!
\end{equation*}
Таким образом доказано, что погрешность интерполирования можно представить в виде
\begin{equation*}
    f(x)-L_n(x) = \frac{f^{(n+1)}(\xi)}{(n+1)!}\omega(x),
\end{equation*}
где $\xi \in [a,b]$ и $\omega(x)$ - многочлен, определенный согласно (\ref{вспомогательный многочлен}).
Отсюда следует оценка
\begin{equation*}
    |f(x)-L_n(x)| \leq \frac{M_{n+1}}{(n+1)!} |\omega(x)|,
\end{equation*}
где $M_{n+1} = \sup_{x \in [a,b]}|f^{(n+1)}(x)|.$ В частности, если $f(x)$ - алгебраический многочлен степени $n$, то интерполирование, проведенное по любым точкам $x_0, x_1, ..., x_n$, осуществляется точно, т.е. $L_n(x) \equiv f(x).$


% -------- source --------
\bigbreak
[\cite[page 127-129, 132-133]{chm_samarski_gulin}]
